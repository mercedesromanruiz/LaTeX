\chapter{Combustibles fósiles}
\section{Introducción}

Podemos observar que el porcentaje de derivados de petróleo se ha disminuido. En España podemos decir que utilizamos poco carbón, hemos sustituido el carbón de las calderas de las casas por gas natural. Sigue siendo la principal fuente de energía de nuestra sociedad.

\section{Petróleo}

En el crudo podemos encontrar hidrocarburos (alcanos, cicloalcanos, hidrocarburos aromáticos...), compuestos orgánicos con nitrógeno, oxígeno, azufre..., trazas de metales: hirro, vanadio...

Podemos clasificar el crudo por:
\begin{itemize}
    \item Localización
    \item Contenido de Azufre (Dulce ó Agrio)
    \item Gravedad API (liviano, mediano, pesado).
    \begin{equation}
        gravedad\ API = \frac{141,5}{GE(60^\circ F)} - 131,5 
    \end{equation}
    Un crudo más denso que el agua tendrá una gravedad $API > 10 ^\circ$ y 
\end{itemize}

Una primera clasificación es de donde viene, su contenido de azufre

\subsection{Reservas}

Llamamos reservas a yacimientos conocidad que son explotables. Estas cambian si se encuentran nuevos yacimientos o si se desarrollan nuevas tecnologías que hacen que yacimientos ya conocidos pasen a ser económicamente explotables. Las reservas de petróleo han aumentado en los últimos años porque se invierten muchos recursos en buscar nuevos yacimientos y en desarrollar nuevas tecnologías para la explotación. Los países con más reservas de petróleo son Venezuela y Arabia Saudí. 

Gracias a la fracturación hidráulica, EEUU y Canadá, pasaron de depender de paises de occidente a ser el mayor productor de petróleo. La fracturación hidráulica consiste en la inyección a alta presión de un fluido (agua, arena y agentes espesantes) para agrietar formaciones rocosas facilitando el flujo del crudo y gas natural. 

Controvertida:
\begin{itemize}
    \item Independncia energética
    \item Sustituir carbón por petróleo y gas natural
    \item Contaminación de aguas
    \item Puede provocar pequeños terremotos
\end{itemize}

\subsection{Crudo: producción y comercio}

El principal productor de crudo es EEUU, el segundo Arabia Saudí y el tercero la federación Rusa. La principal región es Oriente Medio, Norte Amética (EEUU, Canadá y México), Europa (principalmente en Noruega y algo de Reino Unido).

\subsection{Productos derivados: capacidad de regino, producción y comercio}

Una visión ingenua del regino de petróleo: destilación en productos con distinto punto de abullición (distinta longitud de las cadenas de hidrocarburos). Las más cortas tienen puntos de abullición más bajos. Otros procesos químico (cracking) permiten modificaar las proporciones.

\begin{itemize}
    \item Destilados ligeros (GLP, gasolina, nafta)
    \item Destilados medios (keroseno, )
    \item Destilados pesados (fuel oil pesado)
    \item Residuo (parafinas, asfaltos)
\end{itemize}

EEUU a parte de terner mucha capacidad de producción de crudo, también la tiene de refinarlo. Hay muchisima capacidad de refino en Europa, España por ejemplos solo está detras de Italia y Alemania. Donde más capacidad de refino hay es en Asia Pacífico (principalmente China).

\subsection{Algunos datos (España)}

En España tenemos una infraestructura bastante importante, tanto de puertos que pueden recibir, como de oleoductos para almacenar. Los principales países de donde se importa crudo en España son: EEUU, México , Brasil, Nigeria, Angola, Argelia. El mayor consumo de derivados del petróleo está en el transporte.

\section{Gas}

Al igual que el petróleo, el gas natural también es una mezcla de hidrocarburos, mayoritariamente metano.

\subsection{Reservas}

\subsection{Gas naturas (e hidrógeno): producción y comercio}

El mayor productor de Gas natural es EEUU (al igual que de pretróleo). Países como la federación rusa también producen bastante (en torno a la mitad de EEUU), Arabia Saudí, China, Noruega. 

La mayor producción de hidrógeno es a partir de gas natural. El hidrógeno no es una fuente de energía, es un vector energético, como la electricidad. 

\subsubsection{GNL y gasoductos}

La menera de destibuir un gas a cortas distancia es con tuberías (gaseoductos). Para grandes distancias utilizamos Gan Natural Licuado. 

\subsection{Algunos datos (España)}

En España tenemos una infraestructura bastante importante de gas natural, se optó por esto a principios de este siglo. El principal proveedor de Gas Natural es Argelia, luego EEUU y Rusia. 

\subsection{Emisiones de metano}

\section{Carbón}

Podemo clasificar el carbón según su contenido de carbono, humedad y poder calorífico.

\begin{itemize}
    \item Lignito tiene unos contenidos de carbono más bajo que el resto y más humedad
    \item Sub-bitu
    \item Hard Coal 
    \begin{itemize}
        \item Hulla
        \item Antracita
    \end{itemize}
\end{itemize}

Para generar energái eléctrica por ejemplo se utilizan los que tienen menor contenido de carbono. 

\section{Emisionesde $CO_2$ y mitigación }