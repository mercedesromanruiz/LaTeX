\documentclass[main]{subfiles}

\begin{document}
\newpage
\thispagestyle{empty}
\begin{center}
    {\includegraphics[width=0.5\textwidth]{Imagenes/Logo UMA.jpg}\par}
    \vspace{1cm}
    {\bfseries\LARGE \Facultad \par}
    \vspace{0.5cm}
    {\scshape\Large \Grado \par}
    \vspace{3cm}
    {\scshape\Huge Anexos \par}
    \vspace{1.5cm}
    {\itshape\Large \TituloProyecto \par}
    \vfill
    {\Large Solicitante: \par}
    {\Large \Solicitante  \par}
    \vspace{1cm}
    {\Large Autores: \par}
    {\Large \Autora \par}
    {\Large \Autor \par}
    \vfill
    {\Large \Fecha \par}
\end{center}

% El documento básico Anexos se iniciará con un índice que hará referencia a cada uno de los documentos, a sus capítulos y apartados que los componen, con el fin de facilitar su utilización. 
\chapter*{Índice de Anexos:}
\tableof{Anexos}
\newpage
\toftagstart{Anexos}

% Está formado por los documentos que desarrollan, justifican o aclaran apartados específicos de la memoria u otros documentos básicos del Proyecto. Este documento contendrá los anexos necesarios.
\addcontentsline{toc}{chapter}{Anexo I: Fontanería}
\subfile{Anexos/Anexo I Instalacion de Fontaneria}

\addcontentsline{toc}{chapter}{Anexo II: Saneamiento}
\subfile{Anexos/Anexo II Saneamiento}

\addcontentsline{toc}{chapter}{Anexo III: Iluminación}
\subfile{Anexos/Anexo III Iluminacion}

\addcontentsline{toc}{chapter}{Anexo IV: Electricidad}
\subfile{Anexos/Anexo IV Instalacion Electrica}

% Otros anexos
\toftagstop{Anexos}
\end{document}