\documentclass[../main.tex]{subfiles}

\begin{document}
\newpage
\thispagestyle{empty}
\begin{center}
    {\includegraphics[width=0.5\textwidth]{Imagenes/Logo UMA.jpg}\par}
    \vspace{1cm}
    {\bfseries\LARGE \Facultad \par}
    \vspace{0.5cm}
    {\scshape\Large \Grado \par}
    \vspace{3cm}
    {\scshape\Huge Pliego de Condiciones \par}
    \vspace{1.5cm}
    {\itshape\Large \TituloProyecto \par}
    \vfill
    {\Large Solicitante: \par}
    {\Large \Solicitante  \par}
    \vspace{1cm}
    {\Large Autores: \par}
    {\Large \Autora \par}
    {\Large \Autor \par}
    \vfill
    {\Large \Fecha \par}
\end{center}

% El Pliego de Condiciones se iniciará con un índice que hará referencia a cada uno de los documentos, a sus capítulos y apartados que los componen, con el fin de facilitar su utilización
\chapter*{Índice del Pliego de Condiciones:}
\tableof{Pliego}
\newpage
\toftagstart{Pliego}

\section{Introducción y descripción del proyecto}
\subsection{Objeto del Pliego}
El presente pliego de condiciones tiene por objeto fijar las condiciones particulares de los materiales, métodos y equipos de trabajo del Proyecto de Carpintería de Madera, así como la enumeración de la normativa legal a las que se ha de ajustar la obra en cuestión, para la ejecución del Proyecto que se complementa con las especificaciones técnicas incluidas en cada anexo de la memoria descriptiva. \par
\vspace{0.5 cm}
Además se establece en el presente pliego los criterios y medios con los que se pueden estimar y valorar las obras a realizar, así como el periodo de ejecución, la fecha de inicio y de recepción de la obra.

\subsection{Documentos del proyecto}
Los documentos que quedan incorporados al Contrato como documentos contractuales, son los siguientes:
\begin{itemize}
    \item Memoria descriptiva:Justifican las soluciones adoptadas en las diversas instalaciones que conforman el proyecto.
    \item Planos: Define técnicamente y geométricamente el proyecto.
    \item Pliego de Prescripciones Técnicas Particulares: Especifica las características que han de tener la ejecución de las instalaciones y los materiales a emplear. 
    \item Presupuesto: Constituye un presupuesto general orientativo que servirá para el pago de las obrar realizadas.
    \item Estudio Básico de Seguridad y Salud: Define las medidas preventivas adecuadas a los riesgos que conlleva la realización de la obra.
\end{itemize}
Cabe destacar que las obras que se llevarán a cabo, estarán destinadas a la ejecución de las instalaciones de una nave destinada a carpintería de madera. Las instalaciones a realizar son las correspondientes a fontanería, saneamiento, iluminación, baja tensión, fotovoltaica y de protección contra incendios.

\section{Condiciones generales}
En este capítulo se especifican con claridad las condiciones de índole facultativa, económica y legal que regirán en el desarrollo de las obras. Se pueden considerar los siguientes apartados:
\subsection{Condiciones generales facultativas}
Se exigirá al Propietario una fianza del $5 \%$ del presupuesto de ejecución de las obras contratadas que se fije en el Contrato, que le será devuelto una vez finalizado el plazo de garantía, previo informe favorable de la Dirección Facultativa. \par
\vspace{0.5 cm}
Toda la obra se ejecutará con estricta sujeción al proyecto que sirve de base a la Contrata, a este Pliego de Condiciones y a las ordenes e instrucciones que se dicten por el Director de obra. El orden de los trabajos será fijado por ellos, señalándose los plazos prudenciales para la buena marcha de las obras. \par
\vspace{0.5 cm}
El Propietario deberá abonar el importe de todos los trabajos ejecutados, previa medición realizada conjuntamente por éste y la Dirección Facultativa, siempre que aquellos se hayan realizado de acuerdo con el Proyecto y las Condiciones Generales y Particulares que rijan en la ejecución de la obra. \par
\vspace{0.5 cm}
El precio de contrata es el que comprende el coste total de obra.

\subsubsection{Obras que comprende el proyecto}
Las Obras regladas por el presente Pliego están descritas en la Memoria y definidas en los Planos y demás documentos del Proyecto. \par
\vspace{0.5 cm}
Las disposiciones de carácter general de este Pliego quedarán asimismo vigentes para las unidades de obra que, como consecuencia de nuevas necesidades, imprevistos o modificaciones del Proyecto, fuese necesario ejecutar y no estuvieran incluidas en los documentos del mismo.

\subsubsection{Inicio de obras}
El adjudicatario deberá dar comienzo a las obras dentro de los quince días siguientes a la fecha de la adjudicación definitiva a su favor, dando cuenta de oficio a la Dirección Técnica, del día que se propone inaugurar los trabajos, quien acusará recibo. \par
\vspace{0.5 cm}
Las obras deberán quedar total y absolutamente terminadas en el plazo que se fije en la adjudicación a contar desde igual fecha que en el caso anterior. No se considerará motivo de demora de las obras la posible falta de mano de obra o dificultades en la entrega de materiales.

\subsubsection{Términos de recepción}
El director de la obra comunicará a la propiedad de la proximidad de su terminación, para acordar la fecha para el acto de recepción provisional. Ésta se realizará con la intervención de un técnico designado por la propiedad del Constructor y del director de la obra. También se convocará a los restantes técnicos que, en su caso, hubiesen intervenido en la dirección con función propia en aspecto parciales o unidades especializadas. Desde esta fecha comenzará el plazo de garantía si la obra se hallase en estado de ser admitida, y seguidamente con los técnicos de la dirección facultativa extenderán el correspondiente Certificado Final de Obra. Al realizarse la recepción provisional de la obra, deberá presentar el contratista las pertinentes autorizaciones de los organismos oficiales de la provincia para el uso y puesta en marcha de la instalación que así lo requiera. \par
\vspace{0.5 cm}
Si se encuentran las obras ejecutadas en buen estado y con arreglo a las prescripciones previstas, la Dirección Facultativa las dará por recibidas y se entregarán al uso de la propiedad, tras la firma de la correspondiente acta. Cuando las obras no se hallen en estado de ser recibidas se hará constar así en el acta y el director de las mismas señalará los defectos observados y detallará las instrucciones precisas, fijando un plazo para remediar aquéllos. Si transcurrido dicho plazo el contratista no lo hubiera efectuado, podrá concederse otro nuevo plazo improrrogable o declarar resuelto el contrato.

\subsubsection{Criterios de medición}
\begin{itemize}
    \item Partidas. Se seguirán los mismos criterios que figuran en las hojas de estado de mediciones.
    \item Partidas no contenidas. Se efectuará su medición, salvo pacto en contrario, según figura en el Pliego General de Condiciones.
    \item Partidas alzadas. Su precio se fijará a partir de la medición correspondiente y precio contratado o con la justificación de mano de obra y materiales utilizados.
\end{itemize}

\subsection{Condiciones generales económicas}
Este apartado contiene la descripción completa de condiciones económicas acordadas entre la Propiedad y la Contrata, así como regula las funciones de control económico asignadas  a la Dirección Facultativa.

\subsubsection{Garantía o Fianza. Concepto y Condiciones de devolución}
La Fianza es una cantidad monetaria, generalmente un porcentaje sobre del montante global, que debe depositar la Contrata en el momento de la firma del contrato, en concepto de garantía. En caso necesario, se cargarían a esta fianza:
\begin{itemize}
    \item Toda penalización por fallo ó demora.
    \item Toda reparación que por causa de la ejecución, tenga que abonar la Propiedad, y sean con cargo a la Contrata.
    \item La finalización unilateral (abandono) de la ejecución por el Contratista.
\end{itemize}

\subsubsection{Precios contratados}
Se ajustarán a los proporcionados por el Contratista en la oferta.

\subsubsection{Precios contradictorios}
De acuerdo con el Pliego General de Condiciones, aquellos precios de trabajos que no figuren entre los contratados, se fijarán contradictoriamente entre la Dirección Facultativa y el Propietario, presentándolos éste de modo descompuesto y siendo necesaria su aprobación para la posterior ejecución en obra.

\subsubsection{Indemnizaciones por retraso}
El importe de la indemnización por retaso no justificado en el plazo de terminación de la obra se establecerá en un tanto por mil del importe total de los trabajos contratados, por cada día natural de retardo, contados a partir del día de terminación fijado en el calendario de obra. Este tanto por mil será aprobado entre las partes del Propietario, Dirección Facultativa y Contrata.

\subsubsection{Revisiones de precios}
Habrá lugar a revisión de precios cuando así lo contemple el Contrato suscrito entre la Propiedad y el Contratista.

\subsubsection{Mejoras y Modificados de obra, instalaciones y maquinaria}
Se ordena el sistema de valoración de las mejoras o modificaciones que van surgiendo y que se produzca por deseo de la Propiedad, sugerencia de la Dirección (y aprobado por la propiedad) o por necesidad constructiva. Se especifica además que las mejoras propuestas por la Contrata no generen un aumento del importe de las obras o instalaciones.

\subsubsection{Valoración y abono de trabajos}
Según la modalidad elegida para la contratación de la obra u salvo que el pliego particular de condiciones económicas se acuerde otra cosa, pudiéndose efectuar dicho abono de la siguiente forma:
\begin{itemize}
    \item Tipo fijo o tanto alzado total. Se abonará la cantidad previamente fijada como base de la adjudicación, disminuida en su caso en el importe de la baja ejecutada por el adjudicatario.
    \item Tipo dijo o tanto alzado por unidad de obra. Se abonará la cantidad fijada de antemano, pudiendo variar únicamente el número de unidades de obra.
    \item Tanto variable por unidad de obra, según las condiciones en que se realice y los materiales empleados en su ejecución de acuerdo con las órdenes del director técnico.
    \item Por lista de jornales y recibos de materiales, autorizados en la forma que el presente pliego de condiciones determina.
    \item Por horas de trabajo, ejecutado en las condiciones determinadas en el contrato.
\end{itemize}
El criterio elegido será redactado y firmado entre el Propietario, Dirección Facultativa y Contrata.

\subsubsection{Seguros y conservación de la obra, maquinaria e instalaciones}
Se obliga a la Contrata a suscribir los seguros necesarios para garantizarlos posibles daños en las obras, maquinaria e instalaciones que se produjeran por causa de las propias obras. En concreto, se exige a la Contrata la suscripción de un Seguro de Responsabilidad Civil. Asimismo, se exige a la Contrata que establezca los medios precisos para evitar los robos o daños producidos por terceras personas.  

\subsubsection{Condiciones de pago}
El abono de las cantidades a pagar por maquinaria, equipos e instalaciones se rige de forma diferente al que se pacta para la construcción, en especial cuando se contrata con otros suministradores directamente. Una posible forma de pago puede ser: 
\begin{itemize}
    \item un $10\%$ a la firma del contrato. 
    \item uno o varios abonos para la entrega de los materiales o equipos en obra y para las distintas fases de montaje.
    \item un $10-20\%$ a la puesta en marcha. 
    \item  un $10\%$ a los 6 o 12 meses de la puesta en marcha (periodo de garantía).
\end{itemize}
Si la intención de la Propiedad es la contratar la maquinaria y equipos mediante operaciones especiales, como el leasing, es conveniente hacer referencia a ello. 

\subsection{Condiciones generales legales}

\subsubsection{Obligaciones y responsabilidades de la dirección técnica}
\begin{itemize}
    \item Trabajos defectuosos. \par
    \vspace{0.5 cm}
    En el caso de que el Director de la obra encontrase razones fundadas para creer en la existencia de defectos en la obra ejecutada, ordenará efectuar, en cualquier momento y previo a la recepción definitiva, las demoliciones que crea necesarias para el reconocimiento de aquellos.
    \item Inalterabilidad del proyecto. \par
    \vspace{0.5 cm}
    El proyecto (y anexos si los hubiera) será inalterable salvo que la dirección técnica renuncia expresamente a dicho proyecto, o fuera rescindido el convenio de prestación de servicios, en los términos y condiciones legalmente establecidos.
    \item Inspección y medidas previas al montaje. \par
    \vspace{0.5 cm}
    Antes de comenzar los trabajos de montaje, la empresa instaladora deberá efectuar el replanteo de todos y cada uno de los elementos de la instalación, equipos, aparatos y conductores. En caso de discrepancias entre las medidas realizadas en obra y las que aparecen en los planos, que impidan la correcta realización de los trabajos de acuerdo a la normativa vigente, la empresa instaladora deberá notificar las anomalías a la dirección de obra para las oportunas rectificaciones.
\end{itemize}

\subsubsection{Obligaciones y responsabilidades del Contratista}
\begin{itemize}
    \item Definición. \par
    \vspace{0.5 cm}
    Se entiende por contratista la parte contratante obligada a ejecutar la obra. El Contratista estará obligado a redactar un plan completo de Seguridad e higiene específico para la presente obra, conformado y que cumplan las disposiciones vigentes, no eximiéndole el incumplimiento o los defectos del mismo de las responsabilidades de todo género que se deriven. Dicho plan será acordado por el Coordinador de Seguridad y Salud. \par
    \vspace{0.5 cm}
    En caso de accidentes ocurridos a los operarios, en el transcurso de ejecución de los trabajos de la obra, el Contratista se atenderá a lo dispuesto a este respecto en la legislación vigente, siendo en todo caso, único responsable de su incumplimiento y sin que por ningún concepto pueda quedar afectada la Propiedad ni la Dirección Facultativa, por responsabilidad en cualquier aspecto. \\
    El Contratista será responsable de todos los accidentes que por inexperiencia o descuido sobrevinieran, tanto en la propia obra como en las edificaciones contiguas. Será por tanto de su cuenta el abono de las indemnizaciones a quien corresponda y, de todos los daños y perjuicios que puedan causarse en los trabajo de ejecución de la obra, cuando a ello hubiera lugar (todo ello en base a la legislación vigente). \par
    \vspace{0.5 cm}
    La Normativa de obligado cumplimiento para el Contratista queda contemplada en el último apartado de esta parte del Pliego.
    \item Personal. \par
    \vspace{0.5 cm}
    El nivel técnico y la experiencia del personal aportado por el contratista serán adecuados, en cada caso, a las funciones que le hayan sido encomendadas.
    \item Conocimiento y modificación del proyecto. \par
    \vspace{0.5 cm}
    El contratista deberá conocer el Proyecto en todos sus documentos, solicitando en cada caso necesario todas las aclaraciones que estime oportunas para la correcta interpretación de los mismos en la ejecución de la obra. Podrá proponer todas las modificaciones constructivas que crea adecuadas a la consideración del Director de obra, pudiendo llevarlas a cabo con la autorización por escrito de éste.
    \item Oficina en la obra. \par
    \vspace{0.5 cm}
    El Constructor habilitará en la obra una oficina en la que existirá una mesa o tablero adecuado, en el que se puedan consultar los planos. En dicha oficina tendrá siempre el Contratista a disposición de la Dirección Facultativa:
    \begin{itemize}
        \item El Proyecto de Ejecución completo.
        \item La Licencia de obras.
        \item El Libro de Ordenes y Asistencias.
        \item El Plan de Seguridad e Higiene.
        \item El Libro de Incidencias.
        \item El Reglamento y Ordenanza de Seguridad e Higiene en el Trabajo.
    \end{itemize}
    Dispondrá además el Constructor de una oficina para la Dirección Facultativa, convenientemente acondicionada para que en ella se pueda trabajar con normalidad a cualquier hora de la jornada.
    \item Replanteo. \par
    \vspace{0.5 cm}
    El Constructor (u otro) iniciará las obras con el replanteo de las mismas en el terreno, señalando las referencias principales que mantendrá como base de posteriores replanteos parciales. Dichos trabajos se incluirán dentro de la oferta del contratista. \par
    \vspace{0.5 cm}
    El Constructor someterá el replanteo a la aprobación del director técnico, una vez que este haya dado su conformidad, éste preparará un acta acompañada de un plano que deberá ser aprobada pro el director técnico.
    \item Responsabilidades. \par
    \vspace{0.5 cm}
    El contratista es el único responsable de la ejecución de los trabajos que ha contratado y, pro consiguiente, de los defectos que, bien por la mala ejecución o por la deficiente calidad de los materiales empleados, pudieran existir. También será responsable de aquellas partes de la obra que subcontrate, siempre con constructores legalmente capacitados.
    \item Materiales y equipo. \par
    \vspace{0.5 cm}
    El contratista aportará los materiales y medios auxiliares necesarios para la ejecución de la obra en su debido orden de trabajos. Estará obligado a realizar con sus medios, materiales y personal, cuanto disponga la Dirección Facultativa en orden a la seguridad y buena marcha de la obra.
    \item Limpieza de la obra. \par
    \vspace{0.5 cm}
    Es obligación del Constructor u otro mantener limpias las obras y sus alrededores, tanto de escombros como de materiales sobrantes, hacer desaparecer las instalaciones provisionales que no sean necesarias, así como adoptar las medidas y ejecutar todos los trabajos que sean necesarios para que la obra ofrezca un buen aspecto.
\end{itemize}

\subsubsection{Obligaciones y responsabilidades del Coordinador de Seguridad y Salud}
\begin{itemize}
    \item Seguridad e higiene en la obra. \par
    \vspace{0.5 cm}
    El Contratista asumirá las responsabilidades de Coordinador de Seguridad y Salud, cuidando que las obras se realicen de acuerdo a las prescripciones establecidas en la Ley 31/95 y reglamentos que la desarrollan. \par
    \vspace{0.5 cm}
    Asimismo, el Contratista será el responsable de los accidentes que pudieran producirse en el desarrollo de la obra por impericia o descuido, y de los daños que por la misma causa pueda ocasionar a terceros. \par
    \vspace{0.5 cm}
    En el caso de que por simplicidad de la obra no aparezca la figura del Contratista, asumirá el citado cargo el Director de la obra.
\end{itemize}

\subsubsection{Obligaciones y responsabilidades del Propietario}
\begin{itemize}
    \item Desarrollo técnico. \par
    \vspace{0.5 cm}
    La Propiedad podrá exigir de la Dirección Facultativa el desarrollo técnico adecuado del Proyecto y de su ejecución material, dento de las limitaciones legales existentes.
    \item Personal \par
    \vspace{0.5 cm}
    El nivel técnico y la experiencia del personal aportado por el contratista serán adecuados, en cada caso, a las funciones que le hayan sido encomendadas.
    \item Interrupción de las obras. \par
    \vspace{0.5 cm}
    La Propiedad podrá desistir en cualquier momento de la ejecución de las obras de acuerdo con lo que establece el Código Civil, sin perjuicio de las indemnizaciones que, en su caso, deba satisfacer.
    \item Cumplimiento de la normativa urbanística. \par
    \vspace{0.5 cm}
    De acuerdo con lo establecido por la ley sobre Régimen del Suelo y Ordenación Urbana, la propiedad estará obligada al cumplimiento de todas las disposiciones sobre ordenación urbana vigentes, no pudiendo comenzarse las obras sin tener concedida la correspondiente licencia de los organismos competentes. Deberá comunicar a la Dirección Facultativa dicha concesión, pues de los contrario, ésta podrá paralizar las obras, siendo la Propiedad la única responsable de los perjuicios que pudieran derivarse.
    \item Actuación en la ejecución de la obra. \par
    \vspace{0.5 cm}
    La Propiedad se abstendrá de ordenar la ejecución de obra alguna o la introducción de modificaciones sin la autorización de la Dirección Facultativa, así como a dar a la Obra un uso distinto para el que fue proyectada, dado que dicha modificación pudiera afectar a la seguridad del edificio por no estar prevista en las condiciones de encargo del Proyecto.
    \item Honorarios. \par
    \vspace{0.5 cm}
    El propietario está obligado a satisfacer en el momento oportuno todos los honorarios que se hayan contratado con la Dirección Facultativa.
\end{itemize}
\section{Condiciones técnicas particulares}
Este Pliego de Condiciones Técnicas Particulares, el cual forma parte de la documentación del presente proyecto y que regirá las obras e instalaciones para la realización del mismo, tiene como misión estableces las condiciones técnicas y legales para que el objeto del Proyecto pueda materializarse en las condiciones especificadas, evitando posibles interpretaciones diferentes de las deseadas.

\subsection{Suministro de agua}
\subsubsection{Generalidades}

\subsubsection{Reglamentos de aplicación}
\begin{itemize}
    \item Real Decreto 3/2023 de 10/01/2023, por el que se establecen los criterios técnico-sanitarios de la calidad del agua de consumo, su control y suministro.
    \item Real Decreto 487/2022 de 21/06/2022, por el que se establecen los requisitos sanitarios para la prevención y el control de la legionelosis
    \item Decreto 327/2012 de 10/07/2012, por el que se modifican diversos Decretos para su adaptación a la normativa estatal de transposición de la Directiva de Servicios.
    \item Decreto 9/2011 de 18/01/2011, por el que se modifican diversas Normas Reguladoras de Procedimientos Administrativos de Industria y Energía
    \item Real Decreto 314/2006 de 17/03/2006, por el que se aprueba el Código Técnico de la Edificación
    \item Corrección, de errores y erratas de la Orden VIV/984/2009, de 15 de abril, por la que se modifican determinados documentos básicos del Código Técnico de la Edificación, aprobados por el Real Decreto 314/2006, de 17 de marzo, y el Real Decreto 1371/2007, de 19 de octubre.
    \item Orden 984/2009 de 15/04/2009, por la que se modifican determinados documentos básicos del Código Técnico de la Edificación aprobados por el Real Decreto 314/2006, de 17 de marzo, y el Real Decreto 1371/2007, de 19 de octubre.
    \item Real Decreto 1371/2007 de 19/10/2007, por el que se aprueba el documento básico «DB-HR Protección frente al ruido» del Código Técnico de la Edificación y se modifica el Real Decreto 314/2006, de 17 de marzo, por el que se aprueba el Código Técnico de la Edificación.
    \item Real Decreto 314/2006 de 17/03/2006, por el que se aprueba el Código Técnico de la Edificación
\end{itemize}

\subsubsection{Montaje, utilización y conservación}
\subsubsection{Señalización}
La caracterización de los diferentes fluidos que circulan por las tuberías se realizará mediante colores según los siguientes criterios:
\begin{itemize}
    \item Un color básico: que definirá el fluido que conduce la tubería.
    \item Un color accesorio: que definirá las condiciones y estado del fluido que conduce.
    \item Un signo de peligro: que se empleara en cualquiera de los dos casos anteriores, cuando sea necesario indicar la presencia de peligro proveniente del estado o naturaleza del fluido.
\end{itemize}
Los colores básicos normalizados serán según norma UNE 1063, que a grandes rasgos y sin que sirva para causar errores de definición y solo como ayuda (para una elección correcta utilizar norma UNE 1063), se expresan a continuación:
\begin{itemize}
    \item Agua: Verde
    \item Vapor: Morado
    \item Aire: Violeta
    \item Gas Combustible: Amarillo
    \item Productos Químicos: Gris
    \item Aceites Combustibles y Lubricantes: Marrón
    \item Productos No Especificados: Negro
\end{itemize}
Los colores accesorios se plasman en las conducciones mediante la adición de anillos coloreados sobre el color básico. Dado el numero creciente de tipos de fluidos transportados por las tuberías, no es posible la normalización de todos los casos, no obstante expresamos dos casos:
\begin{itemize}
    \item Anillo color bermellón: para caracterizar la tuberías que transportan fluidos destinados a combatir incendios.
    \item Anillo azul sobre fondo básico verde, para caracterizar las tuberías que transportan agua potable.
\end{itemize}
El signo de peligro estará constituido por una anillo anaranjado con bordes negros pintado sobre color básico, e indicara que el fluido transportado es peligroso.

\subsubsection{Inspecciones y pruebas}
\begin{itemize}
    \item Inspecciones. \par
    \vspace{0.5 cm}
    Antes de iniciarse el funcionamiento de las instalaciones las Empresas o personas instaladoras estarán obligadas a realizar las pruebas de resistencia mecánica y estanqueidad previstas en el apartado 6.2.2.1 del titulo 6º de las Normas Básicas para las Instalaciones Interiores de Agua, para lo cual deberán dar cuenta de ello a la Delegación Provincial del Ministerio de Industria. \par
    \vspace{0.5 cm}
    Si la Delegación no considera necesaria su presencia, facultará al instalador para que, con el usuario o propietario, realice las pruebas. \par
    \vspace{0.5 cm}
    Efectuadas las pruebas previstas en estas Normas Básicas, con o sin la presencia de representantes de la Delegación Provincial del Ministerio de Industria, se procederá a levantar certificado del resultado, que deberá ser suscrito, al menos, por el usuario o propietario y la Empresa Instaladora. Copia de este certificado, deberá enviarse a la Delegación Provincial del Ministerio de Industria. \par
    Se entenderá que las instalaciones tendrán la aprobación de funcionamiento por la Delegación Provincial del Ministerio de Industria si, transcurridos treinta días desde el envío de la copia del certificado, la Delegación Provincial del Ministerio de Industria no manifiesta objeción alguna al respecto. \par
    \vspace{0.5 cm}
    Los Servicios Técnicos del a Delegación Provincial del Ministerio de Industria podrán realizar en las instalaciones las pruebas reglamentarias y efectuar las inspecciones, supervisiones y comprobaciones que consideren necesarias para asegurar el buen funcionamiento de las instalaciones. \par 
    \vspace{0.5 cm}
    \item Pruebas. \par
    \vspace{0.5 cm}
    Todos los elementos y accesorios que integran las instalaciones serán objeto de las pruebas reglamentarias. \par
    \vspace{0.5 cm}
    Antes de proceder al empotramiento de las tuberías, las Empresas instaladoras están obligadas a efectuar la siguiente prueba:
    \begin{enumerate}
        \item Serán objeto de esta prueba todas las tuberías, elementos y accesorios que integran la instalación. 
        \item La prueba se efectuará a 20 Kg/cm2. Para iniciar la prueba se llenará de agua toda la instalación manteniendo abiertos los grifos terminales hasta que se tenga la seguridad de que la purga ha sido completa y no queda nada de aire. Entonces se cerrarán todos los grifos que nos han servido de purga y el de la fuente de alimentación. A continuación se empleara la bomba, que ya estar conectada y se mantendrá su funcionamiento hasta alcanzar la presión de prueba. Una vez conseguida, se cerrará la llave de paso de la bomba. Se procederá a reconocer toda la instalación para asegurarse de que no existe perdida.
        \item A continuación se disminuirá la presión hasta llegar a la de servicio, con un mínimo de 6 Kg/cm2 y se mantendrá esta presión durante quince minutos. Se dará por buena la instalación si durante este tiempo la lectura del manómetro ha permanecido contante. \\
        El manómetro a emplear en esta prueba deberá apreciar, con claridad, décimas de Kg/cm2. 
        \item Las presiones aludidas anteriormente se refieren a nivel de la calzada.
    \end{enumerate}
\end{itemize}

\subsection{Saneamiento}
\subsubsection{Normativa de aplicación}
La redes de alcantarillado se diseñarán y construirán de acuerdo con lo que establece la siguiente normativa:
\begin{itemize}
    \item Obligatoria: \\
    La redes de alcantarillado se diseñarán y construirán de acuerdo con lo que establece la siguiente normativa:
    \begin{itemize}
        \item ORDEN del MOPU del 15-09-86 Pliego de Prescripciones Técnicas de tuberías de saneamiento de poblaciones.
        \item RD 849/1986 por el que se aprueba el Reglamento de Dominio Público Hidráulico. BOE: 30-04-86.
    \end{itemize}
    \item Recomendada:
    \begin{itemize}
        \item ORDEN del Ministerio de la Vivienda del 31-07-73 NTE-ISS: Instalación de evacuación de salubridad: saneamiento del edificio.
        \item ORDEN del Ministerio de la Vivienda del 09-01-74 NTE-ISD: Depuración y vertido de Aguas Residuales.
        \item ORDEN del Ministerio de la Vivienda del 18-04-77 NTE-ASD: Sistemas de Drenajes.
    \end{itemize}
\end{itemize}
También debe tenerse en cuenta para que toda la red de alcantarillado incluidos sus elementos complementarios tenga garantizada la calidad, funcionalidad, durabilidad y rendimiento esperados las Normas UNE que cubren estas exigencia.

\subsubsection{Condiciones del proceso de ejecución de las obras}
Las piezas no se colocarán hasta que se haya comprobado que la superficie sobre la que se asentarán cumple las condiciones de calidad y forma previstas, con las tolerancias establecidas. \par
\vspace{0.5 cm}
Si en esta superficie hay defectos o irregularidades superiores a las tolerables, se corregirán antes de ejecutar la partida de obra. \par
\vspace{0.5 cm}
Antes de bajar las piezas a la zanja, la Dirección Facultativa las examinará, rechazando las que presenten algún defecto perjudicial. \par
\vspace{0.5 cm}
La descarga y la manipulación de las piezas se harán de forma que no sufran golpes. El fondo de la zanja estará limpio antes de bajar las piezas. La colocación de las piezas prefabricadas comenzará por el punto más bajo.

\subsubsection{Control y criterios de aceptación y rechazo}
Se comprobará la rasante de los conductos entre pozos, con un control en un tramo de cada tres.
\begin{itemize}
    \item No se aceptará cuando se produzca una variación en la diferencia de cotas de los pozos extremos superior al 20\%. Se comprobará los recalces y corchetes, con un control cada 15 m.
    \item No se aceptará cuando se produzca una ejecución defectuosa o deficiencia superior a 5cm. Se comprobará la estanqueidad del tramo sometido a una presión de 0,5 ATM con una prueba general.
    \item No se aceptará cuando se produzca una fuga antes de tres horas. Cuando se refuerce la canalización se comprobará el espesor sobre conductos mediante una inspección general.
    \item No se aceptará cuando existan deficiencias superiores al 10\%.
\end{itemize}

\subsubsection{Condiciones de uso y mantenimiento}
No se verterán a la red basuras, ni aguas de las siguientes características:
\begin{itemize}
    \item pH menor que 6 y mayor que 9.
    \item Temperatura superior a 40°C
    \item Conteniendo detergentes no biodegradables.
    \item Conteniendo aceites minerales orgánicos y pesados.
    \item Conteniendo colorantes permanentes y sustancias tóxicas.
    \item Conteniendo una concentración de sulfatos superior a 0,2 g/l.
\end{itemize}

\subsection{Iluminación}
\subsubsection{Luminarias}
Serán de los tipos señalados en la memoria del presente proyecto o equivalentes y cumplirán obligatoriamente las prescripciones fijadas en la Instrucción ITC-BT-44 del REBT y el DB HE-3 del CTE. En cualquier caso serán adecuadas a la potencia de las lámparas a instalar en ellas y cumplirán con lo prescrito en las Normas UNE correspondientes. \par
\vspace{0.5 cm}
Tendrán curvas fotométricas, longitudinales y transversales simétricas respecto a un eje vertical, salvo indicación expresa en sentido contrario en alguno de los documentos del Proyecto o de la Dirección Facultativa. \par
\vspace{0.5 cm}
Su masa no sobrepasará los 5 Kg de peso cuando éstas se encuentren suspendidas excepcionalmente de cables flexibles. \par
\vspace{0.5 cm}
La tensión asignada de los cables utilizados será como mínimo la tensión de alimentación y nunca inferior a 300/300 V siendo necesario que el cableado externo de conexión a la red disponga del adecuado aislamiento eléctrico y térmico. \par
\vspace{0.5 cm}
Las partes metálicas accesibles (partes incluidas dentro del volumen de accesibilidad, ITC-BT-24) luminarias que no sean de Clase I o Clase II deberán tener un elemento de conexión para su puesta a tierra. \par
\vspace{0.5 cm}
De acuerdo con el Documento Básico DB HE-3: Eficiencia energética de las instalaciones de iluminación del Código Técnico de la Edificación (CTE), los edificios deben disponer de instalaciones de iluminación adecuadas a las necesidades de sus usuarios y a la vez eficaces energéticamente, disponiendo de un sistema de control que permita ajustar el encendido a la ocupación real de la zona, así como de un sistema de regulación que optimice el aprovechamiento de la luz natural en las zonas que reúnan determinadas condiciones. Por lo que se deberá de implantar las luminarias y sus controles, según modelos calculados en el presente proyecto, y ubicados en los sitios exactos proyectados, para que el cumplimiento del DB HE-3 sea correcto.

\subsubsection{Instalación de las lámparas}
Las partes metálicas accesibles de los receptores de alumbrado que no sean de Clase II o Clase III, deberán conectarse de manera fiable y permanente al conductor de protección del circuito. \par
\vspace{0.5 cm}
Para instalaciones que alimenten a tubos de descarga con tensiones asignadas de salida comprendidas entre 1kV y 10kV, se aplicará lo dispuesto en la Norma UNE correspondiente. \par
\vspace{0.5 cm}
La protección contra contactos directos e indirectos se realizará, en su caso, según los requisitos de la Instrucción ICT-BT-24 del REBT. \par
\vspace{0.5 cm}
En instalaciones de iluminación que empleen lámparas de descarga donde se ubiquen máquinas rotatorias se adoptarán las precauciones necesarias para evitar accidentes causados por ilusión óptica debida al efecto estroboscópico. \par
\vspace{0.5 cm}
En instalaciones especiales se alimentarán las lámparas portátiles con tensiones de seguridad de 24V, excepto si son
alimentados por medio de transformadores de separación. \par
\vspace{0.5 cm}
Cuando se emplean muy bajas tensiones de alimentación (12 V) se preverá la utilización de transformadores adecuados.
Para los rótulos luminosos y para instalaciones que los alimentan con tensiones asignadas de salida en vacío
comprendidas entre 1 y 10 kV, se aplicará lo dispuesto en la Norma UNE correspondiente.

\subsection{Suministro eléctrico}
En este apartado se establecen las condiciones mínimas aceptables par ala ejecución de Instalaciones Eléctricas Interiores en Baja Tensión, acorde a lo estipulado por el Real Decreto 842/2002 de 2 de agosto por el que se aprueba el Reglamento Electrotécnico para Baja Tensión, el Real Decreto 314/2006, de 17 de marzo, por el que se aprueba el Código Técnico de la Edificación, la Resolución de 05/05/2005, por la que se aprueban las Normas Particulares y Condiciones Técnicas y de Seguridad de la empresa distribuidora de energía eléctrica, Endesa Distribución, SLU, en el ámbito de la Comunidad Autónoma de Andalucía, la Resolución de 23/03/2006, de corrección de errores y erratas de la Resolución de 5 de mayo de 2005, por la que se aprueban las normas particulares y condiciones técnicas y de seguridad de la empresa distribuidora de energía eléctrica, Endesa Distribución SLU, en el ámbito de la Comunidad Autónoma de Andalucía, la Instrucción de 14/10/2004, de la Dirección General de Industria, Energía y Minas, sobre previsión de cargas eléctricas y coeficientes de simultaneidad en áreas de uso residencial y áreas de uso industrial, así como el Real Decreto 1955/2000 de 01/12/2000, ELECTRICIDAD. Regula las actividades de transporte, distribución, comercialización, suministro y procedimientos de autorización de instalaciones de energía eléctrica. \par
\vspace{0.5 cm}
Las dudas que se planteasen en su aplicación o interpretación serán dilucidadas por el técnico designado como Director de Obra. Por el mero hecho de intervenir en la obra, se presupone que la empresa instaladora y las subcontratas conocen y admiten el presente Pliego de Condiciones. \par

\subsubsection{Campo de aplicación}
El presente Pliego de Condiciones Técnicas Particulares se refiere al suministro, instalación, pruebas, ensayos y mantenimiento de materiales necesarios en el montaje de instalaciones eléctricas interiores en Baja Tensión reguladas por el Real Decreto 842/2002 de 2 de agosto por el que se aprueba el Reglamento Electrotécnico para Baja Tensión anteriormente enunciado, con el fin de garantizar la seguridad de las personas, el bienestar social y la protección del medio ambiente, siendo necesario que dichas instalaciones eléctricas se proyecten, construyan, mantengan y conserven de tal forma que se satisfagan los fines básicos de la funcionalidad, es decir de la utilización o adecuación al uso, y de la seguridad, concepto que incluye la seguridad estructural, la seguridad en caso de incendio y la seguridad de utilización, de tal forma que el uso normal de la instalación no suponga ningún riesgo de accidente para las personas y cumpla la finalidad para la cual es diseñada y construida.

\subsubsection{Normativa de aplicación}
Además de las Condiciones Técnicas Particulares contenidas en el presente Pliego, serán de aplicación, y se observarán en todo momento durante la ejecución de la instalación eléctrica interior en BT, las siguientes normas y reglamentos:
\begin{itemize}
    \item Circular de 18/07/2022, sobre locales de pública concurrencia (ITC BT 028)
    \item Real Decreto Ley 29/2021 de 21/12/2021, por el que se adoptan medidas urgentes en el ámbito energético para el fomento de la movilidad eléctrica, el autoconsumo y el despliegue de energías renovables.
    \item Resolución de 29/01/2021, de la Dirección General de Industria y de la Pequeña y Mediana Empresa, por la que se aprueban especificaciones particulares y proyectos tipo de Edistribución Redes Digitales, SLU.
    \item Real Decreto 1183/2020 de 29/12/2020, de acceso y conexión a las redes de transporte y distribución de energía eléctrica.
    \item Real Decreto 542/2020 de 26/05/2020, por el que se modifican y derogan diferentes disposiciones en materia de calidad y seguridad industrial
    \item Resolución de 09/01/2020, de la Dirección General de Industria y de la Pequeña y Mediana Empresa, por la que se actualiza el listado de normas de la instrucción técnica complementaria ITC-BT-02 del Reglamento electrotécnico para baja tensión, aprobado por el Real Decreto 842/2002, de 2 de agosto
    \item Resolución de 14/06/2019, de la Secretaría General de Industria, Energía y Minas, por la que se deroga parcialmente la resolución de 5 de mayo de 2005, de la Dirección General de Industria, Energía y Minas, por la que se aprueban las normas particulares y condiciones técnicas y de seguridad de la empresa distribuidora de energía eléctrica Endesa Distribución, S.L.U., en el ámbito de la Comunidad Autónoma de Andalucía.
    \item Resolución de 05/12/2018, de la Dirección General de Industria y de la Pequeña y Mediana Empresa, por la que se aprueban especificaciones particulares y proyectos tipo de Endesa Distribución Eléctrica, SLU
    \item Reglamento 2016/364 de 01/07/15, relativo a la clasificación de las propiedades de reacción al fuego de los productos de construcción de conformidad con el Reglamento (UE) nº 305/2011 del Parlamento Europeo y del Consejo
    \item Circular de 23/11/2007, instalación de bandejas portacables en locales de pública concurrencia.
    \item Resolución de 23/03/2006, de corrección de errores y erratas de la Resolución de 5 de mayo de 2005, por la que se aprueban las normas particulares y condiciones tecnicas y de seguridad de la empresa distribuidora de energia electrica, Endesa Distribucion SLU, en el ambito de la Comunidad Autonoma de Andalucia
    \item Guía de 01/10/2005, guía técnica de aplicación del reglamento electrotécnico de baja tensión REBT02 (Real Decreto 842/2002). Guía de la ITC BT-24, protección contra contactos directos e indirectos.
    \item Guía de 01/10/2005, guía técnica de aplicación del reglamento electrotécnico de baja tensión REBT02 (Real Decreto 842/2002). Guía de la ITC BT-23, protección contra sobre-tensiones
    \item Guía de 01/10/2005, guía técnica de aplicación del reglamento electrotécnico de baja tensión REBT02 (Real Decreto 842/2002). Guía de la ITC BT-22, protección contra sobre-intensidades.
    \item Guía de 01/10/2005, guía técnica de aplicación del reglamento electrotécnico de baja tensión REBT02 (Real Decreto 842/2002). Guía de la ITC BT-18, instalaciones de puesta a tierra.
    \item Guía de 01/10/2005, guía técnica de aplicación del reglamento electrotécnico de baja tensión REBT02 (Real Decreto 842/2002). Guía de la ITC BT-08, sistemas de conexión del neutro y de las masas en redes de distribución de energía eléctrica.
    \item Guia de 01/10/2005, guia tecnica de aplicacion del reglamento electrotecnico de baja tension REBT02 (Real Decreto 842/2002).
    \item Resolución de 25/10/2005, de la Dirección General de Industria, Energía y Minas, por la que se regula el período transitorio sobre la entrada en vigor de las normas particulares y condiciones técnicas y de seguridad, de Endesa Distribución S.L.U. en el ámbito de esta Comunidad Autónoma
    \item Resolución de 05/05/2005, por la que se aprueban las Normas Particulares y Condiciones Técnicas y de Seguridad de la empresa distribuidora de energía eléctrica, Endesa Distribución, SLU, en el ámbito de la Comunidad Autónoma de Andalucía.
    \item Instrucción de 14/10/2004, de la Dirección General de Industria, Energía y Minas, sobre previsión de cargas eléctricas y coeficientes de simultaneidad en áreas de uso residencial y áreas de uso industrial
    \item Guía de 01/09/2004, guía técnica de aplicación del reglamento electrotécnico de baja tensión REBT02 (Real Decreto 842/2002). Instalaciones de alumbrado exterior (ITC BT 09)
    \item Guía de 01/09/2003, guía técnica de aplicación del reglamento electrotécnico de baja tensión REBT02 (Real Decreto 842/2002). Anexo: verificación de las instalaciones eléctricas
    \item Guía de 01/09/2003, guía técnica de aplicación del reglamento electrotécnico de baja tensión REBT02 (Real Decreto 842/2002). Anexo: calculo de corrientes de cortocircuito
    \item Guía de 01/09/2003, guía técnica de aplicación del reglamento electrotécnico de baja tensión REBT02 (Real Decreto 842/2002). Anexo: calculo de caídas de tensión
    \item Guía de 01/09/2004, guía técnica de aplicación del reglamento electrotécnico de baja tensión REBT02 (Real Decreto 842/2002). Instalaciones en locales de publica concurrencia (ITC BT 028)
    \item Guía de 01/09/2003, guía técnica de aplicación del reglamento electrotécnico de baja tensión REBT02 (Real Decreto 842/2002). Instalaciones eléctricas en muebles (ITC BT 049)
    \item Guía de 01/09/2003, guía técnica de aplicación del reglamento electrotécnico de baja tensión REBT02 (Real Decreto 842/2002). Locales que contienen una bañera o ducha (ITC BT 027)
    \item Guía de 01/09/2003, guía técnica de aplicación del reglamento electrotécnico de baja tensión REBT02 (Real Decreto 842/2002). Instalaciones interiores en viviendas. Prescripciones generales de instalación (ITC BT 026)
    \item Guía de 01/09/2003, guía técnica de aplicación del reglamento electrotécnico de baja tensión REBT02 (Real Decreto 842/2002). Instalaciones interiores en viviendas. Numero de circuitos y características (ITC BT 025)
    \item Guía de 01/09/2003, guía técnica de aplicación del reglamento electrotécnico de baja tensión REBT02 (Real Decreto 842/2002). Tubos y canales protectoras (ITC BT 021)
    \item Guía de 01/09/2003, guía técnica de aplicación del reglamento electrotécnico de baja tensión REBT02 (Real Decreto 842/2002). Instalaciones interiores o receptoras. Sistemas de instalación (ITC BT 020)
    \item Guía de 01/09/2003, guía técnica de aplicación del reglamento electrotécnico de baja tensión REBT02 (Real Decreto 842/2002). Instalaciones interiores o receptoras. Prescripciones generales (ITC BT 019)
    \item Guía de 01/09/2003, guía técnica de aplicación del reglamento electrotécnico de baja tensión REBT02 (Real Decreto 842/2002). Dispositivos generales e individuales de mando y protección. Interruptor de control de potencia (ITC BT 017)
    \item Guía de 01/09/2003, guía técnica de aplicación del reglamento electrotécnico de baja tensión REBT02 (Real Decreto 842/2002). Contadores: ubicación y sistemas de instalación (ITC BT 016)
    \item Guía de 01/09/2003, guía técnica de aplicación del reglamento electrotécnico de baja tensión REBT02 (Real Decreto 842/2002). Derivaciones individuales (ITC BT 015)
    \item Guía de 01/09/2003, guía técnica de aplicación del reglamento electrotécnico de baja tensión REBT02 (Real Decreto 842/2002). Linea general de alimentación (ITC BT 014)
    \item Guía de 01/09/2003, guía técnica de aplicación del reglamento electrotécnico de baja tensión REBT02 (Real Decreto 842/2002). Cajas generales de protección (ITC BT 013)
    \item Guia de 01/09/2003, guia tecnica de aplicacion del reglamento electrotecnico de baja tension REBT02 (Real Decreto 842/2002). Esquemas (ITC BT 012)
    \item Guia de 01/09/2003, guia tecnica de aplicacion del reglamento electrotecnico de baja tension REBT02 (Real Decreto 842/2002). Prevision de cargas para suministros en baja tension (ITC BT 010)
    \item Guía de 01/09/2003, guía técnica de aplicación del reglamento electrotécnico de baja tensión REBT02 (Real Decreto 842/2002). Verificaciones e inspecciones (ITC BT 005)
    \item Guia de 01/09/2003, guia tecnica de aplicacion del reglamento electrotecnico de baja tension REBT02 (Real Decreto 842/2002). Documentacion y puesta en servicio de las instalaciones (ITC BT 004)
    \item Guia de 01/09/2003, guia tecnica de aplicacion del reglamento electrotecnico de baja tension REBT02 (Real Decreto 842/2002). Instaladores autorizados en baja tension (ITC BT 003)
    \item Guia de 01/09/2003, guia tecnica de aplicacion del reglamento electrotecnico de baja tension REBT02 (Real Decreto 842/2002). Real Decreto 842/2002.
    \item Real Decreto 842/2002 de 02/08/2002, por el que se aprueba el Reglamento electrotécnico para baja tensión.
    \item Real Decreto 1955/2000 de 01/12/2000, ELECTRICIDAD. Regula las actividades de transporte, distribución, comercialización, suministro y procedimientos de autorización de instalaciones de energía eléctrica.
\end{itemize}

\subsubsection{Ejecución o montaje de la instalación}
Las instalaciones eléctricas de Baja Tensión serán ejecutadas por instaladores eléctricos habilitados, para el ejercicio de esta actividad, según Instrucciones Técnicas Complementarias ITC del REBT, y deberán realizarse conforme a lo que establece el presente Pliego de Condiciones Técnicas Particulares y a la reglamentación vigente. \par
\vspace{0.5 cm}
La Dirección Facultativa rechazará todas aquellas partes de la instalación que no cumplan los requisitos para ellas exigidas, obligándose la empresa instaladora habilitada o Contratista a sustituirlas a su cargo. \par
\vspace{0.5 cm}
Se cumplirán siempre todas las disposiciones legales que sean de aplicación en materia de seguridad, salud, medio ambiente, eficiencia energética, etc, sean cumplidas por las mismas.

\subsubsection{Preparación del soporte de la instalación eléctrica}
El soporte estará constituido por los parámetros horizontales y verticales, donde la instalación podrá ser vista o empotrada. \par
\vspace{0.5 cm}
En el caso de instalación vista, esta se fijará con tacos y tornillos a paredes y techos, utilizando como aislante protector de los conductores tubos, bandejas o canaletas. \par
\vspace{0.5 cm}
Para la instalación empotrada los tubos flexibles de protección, se dispondrán en el interior de rozas practicadas a los tabiques. \par
\vspace{0.5 cm}
Las rozas no tendrán una profundidad mayor de 4 cm sobre ladrillo macizo y de 1 canuto sobre el ladrillo hueco, el ancho no será superior a dos veces su profundidad. \par
\vspace{0.5 cm}
Las rozas se realizarán preferentemente en las tres hiladas superiores. Si no es así tendrá una longitud máxima de 100 cm.
Cuando se realicen rozas por las dos caras del tabique, la distancia entre rozas paralelas, será de 50 cm. \par
\vspace{0.5 cm}
Se colocarán registros con una distancia máxima de 15 m. Las rozas verticales se separarán de los cercos y premarcos al menos 20 cm y cuando se dispongan rozas por dos caras de paramento la distancia entre dos paralelas será como mínimo de 50 cm, y su profundidad de 4 cm para ladrillo macizo y 1 canuto para ladrillo hueco, el ancho no será superior a dos veces su profundidad. \par
Si el montaje fuera superficial el recorrido de los tubos, de aislante rígido, se sujetará mediante grapas y las uniones de
conductores se realizarán en cajas de derivación igual que en la instalación empotrada. \par
\vspace{0.5 cm}
Se realizará la conexión de los conductores a las regletas, mecanismos y equipos. \par
\vspace{0.5 cm}
Se ejecutará la instalación interior, la cual si es empotrada, se realizarán, rozas siguiendo un recorrido horizontal y vertical y en el interior de las mismas se alojarán los tubos de aislante flexible.

\subsubsection{Comprobaciones iniciales}
Se comprobará que todos los elementos y componentes de la instalación eléctrica de baja tensión, coinciden con su desarrollo en el proyecto, y en caso contrario se redefinirá en presencia de la Dirección Facultativa. Se marcarán, por instalador habilitado en electricidad y en presencia de la Dirección Facultativa, los diversos componentes de la instalación, como tomas de corriente, puntos de luz, canalizaciones, cajas. \par
\vspace{0.5 cm}
Al marcar los tendidos de la instalación se tendrá en cuenta la separación mínima de 30 cm con la instalación de abastecimiento de agua o fontanería. \par
\vspace{0.5 cm}
Se comprobará la situación de la acometida, ejecutada ésta según REBT.

\subsubsection{Características, calidades y condiciones generales de los materiales eléctricos}
\begin{itemize}
    \item Definición y clasificación de las instalaciones eléctricas. \\
    Según Art. 3 del Real Decreto 842/2002, se entiende  por instalación eléctrica todo conjunto de aparatos y de circuitos asociados en previsión de un fin particular: producción, conversión, transformación, transmisión, distribución o utilización de la energía eléctrica. \\
    Asimismo éstas se agrupan y clasifican en:
    \begin{itemize}
        \item Instalación de baja tensión: es aquella instalación eléctrica cuya tensión nominal se encuentra por debajo de 1 kV
        \item Instalación de media tensión: es aquella instalación eléctrica cuya tensión nominal es superior o igual a 1kV e inferior a 66 kV.
        \item Instalación de alta tensión: es aquella instalación eléctrica cuya tensión nominal es igual o superior a 66 kV.
    \end{itemize}
    \item Componentes y productos constituyentes de la instalación. \\
    La instalación proyectada contará con todos los elementos, aparatos y partes de la misma, que se describen en el documento básico Memoria, se detallan en planos y se valoran en el documento básico de Presupuesto.
    \item Control y aceptación de los elementos y equipos que conforman la instalación eléctrica. \\
    El técnico que realice la Dirección técnica del proyecto velará porque todos los materiales, productos, sistemas y equipos que formen parte de la instalación eléctrica sean de marcas de calidad (UNE, EN, CEI, CE, AENOR, etc.) y dispongan de la documentación que acredite que sus características mecánicas y eléctricas se ajustan a la normativa vigente, así como de los certificados de conformidad con las normas UNE, EN, CEI, CE u otras que le sean exigibles por normativa o por prescripción del proyectista y por lo especificado en el presente Pliego de Condiciones Técnicas Particulares. \par
    \vspace{0.5 cm}
    El técnico que realice la Dirección técnica del proyecto asimismo podrá exigir muestras de los materiales a emplear y sus certificados de calidad, ensayos y pruebas de laboratorios, rechazando, retirando, desmontando o reemplazando dentro de cualquiera de las etapas de la instalación los productos, elementos o dispositivos que a su parecer perjudiquen en cualquier grado el aspecto, seguridad o bondad de la obra. \par
    \vspace{0.5 cm}
    Cuando proceda hacer ensayos para la recepción de los productos o verificaciones para el cumplimiento de sus correspondientes exigencias técnicas, según su utilización, estos podrán ser realizadas por muestreo u otro método que indiquen los órganos competentes de las Comunidades Autónomas, además de la comprobación de la documentación de suministro en todos los casos, debiendo aportarse o incluirse, junto con los equipos y materiales, las indicaciones necesarias para su correcta instalación y uso debiendo marcarse con las siguientes indicaciones mínimas: 
    \begin{itemize}
        \item Identificación del fabricante, representante legal o responsable de su comercialización.
        \item Marca y modelo.
        \item Tensión y potencia (o intensidad) asignadas (cuando estos dos factores sean los relevantes).
        \item Cualquier otra indicación referente al uso específico del material o equipo, asignado por el fabricante.
    \end{itemize}
    Concretamente por cada elemento tipo, estas indicaciones para su correcta identificación serán las siguientes:
    \begin{itemize}
        \item Conductores y mecanismos:
        \begin{itemize}
            \item Identificación, según especificaciones de proyecto.
            \item Distintivo de calidad: Marca de Calidad AENOR homologada por el Ministerio de Industria, Comercio y Turismo (MICT).
        \end{itemize}
        \item Contadores y equipos:
        \begin{itemize}
            \item Identificación: según especificaciones de proyecto.
            \item Distintivo de calidad: Tipos homologados por el MICT.
        \end{itemize}
        \item Cuadros generales de distribución:
        \begin{itemize}
            \item Distintivo de calidad: Tipos homologados por el MICT.
        \end{itemize}
        \item Aparatos y pequeño material eléctrico para instalaciones de baja tensión:
        \begin{itemize}
            \item Distintivo de calidad: Marca AENOR homologada por el Ministerio de Industria.
        \end{itemize}
        \item Cables eléctricos, accesorios para cables e hilos para electro-bobinas.
        \begin{itemize}
            \item Distintivo de calidad: Marca AENOR homologada por el MICT.
        \end{itemize}
    \end{itemize}
    El resto de componentes de la instalación deberán recibirse en obra conforme a: la documentación del fabricante, marcado de calidad, la normativa si la hubiere, especificaciones del proyecto y a las indicaciones de la Dirección Facultativa durante la ejecución de las obras. \par
    \vspace{0.5 cm}
    Asimismo aquellos materiales no especificados en el presente proyecto que hayan de ser empleados para la realización del mismo, dispondrán de marca de calidad y no podrán utilizarse sin previo conocimiento y aprobación del técnico que realice la Dirección técnica del proyecto.
\end{itemize}
\subsubsection{Condiciones de mantenimiento y uso}
Las actuaciones de mantenimiento sobre las instalaciones eléctricas interiores de baja tensión son independientes de las inspecciones periódicas que preceptivamente se tengan que realizar. \par
\vspace{0.5 cm}
El titular o la Propiedad de la instalación eléctrica no están autorizados a realizar operaciones de modificación, reparación o mantenimiento. Estas actuaciones deberán ser ejecutadas siempre por una empresa instaladora habilitada. \par
\vspace{0.5 cm}
Durante la vida útil de la instalación, los propietarios y usuarios de las instalaciones eléctricas de generación, transporte, distribución, conexión, enlace y receptoras, deberán mantener permanentemente en buen estado de seguridad y funcionamiento sus instalaciones eléctricas, utilizándolas de acuerdo con sus características funcionales. \par
\vspace{0.5 cm}
La Propiedad o titular de la instalación deberá presentar, junto con la solicitud de puesta en servicio de la instalación que requiera mantenimiento, conforme a lo establecido en las "Instrucciones y Guía sobre la Legalización de Instalaciones Eléctricas de Baja Tensión" (anexo VII del Decreto 141/2009), un contrato de mantenimiento con empresa instaladora habilitada inscrita en el correspondiente registro administrativo, en el que figure expresamente el responsable técnico de mantenimiento. \par
\vspace{0.5 cm}
Los contratos de mantenimiento se formalizarán por períodos anuales, prorrogables por acuerdo de las partes, y en su defecto de manera tácita. Dicho documento consignará los datos identificativos de la instalación afectada, en especial su titular, características eléctricas nominales, localización, descripción de la edificación y todas aquellas otras características especiales dignas de mención. \par
\vspace{0.5 cm}
No obstante, cuando el titular acredite que dispone de medios técnicos y humanos suficientes para efectuar el correcto mantenimiento de sus instalaciones, podrá adquirir la condición de mantenedor de las mismas. En este supuesto, el cumplimiento de la exigencia reglamentaria de mantenimiento quedará justificado mediante la presentación de un Certificado de auto-mantenimiento que identifique al responsable del mismo. No se permitirá la subcontratación del mantenimiento a través de una tercera empresa intermediaria. \par
\vspace{0.5 cm}
Las comprobaciones y chequeos a realizar por los responsables del mantenimiento se efectuarán con la periodicidad acordada, atendiendo al tipo de instalación, su nivel de riesgo y el entorno ambiental, todo ello sin perjuicio de las otras actuaciones que proceda realizar para corrección de anomalías o por exigencia de la reglamentación. Los detalles de las averías o defectos detectados, identificación de los trabajos efectuados, lista de piezas o dispositivos reparados o sustituidos y el resultado de las verificaciones correspondientes deberán quedar registrados en soporte auditable por la Administración. \par
\vspace{0.5 cm}
Para dicho mantenimiento se tomarán las medidas oportunas para garantizar la seguridad del personal. \par
\vspace{0.5 cm}
Las actuaciones de mantenimiento sobre las instalaciones eléctricas son independientes de las inspecciones periódicas que preceptivamente se tengan que realizar.

\subsubsection{Inspecciones periódicas}
Las inspecciones periódicas sobre las instalaciones eléctricas son independientes de las actuaciones de mantenimiento que preceptivamente se tengan que realizar. \par
\vspace{0.5 cm}
Deberán realizarse en los plazos siguientes:
\begin{enumerate}
    \item Serán objeto de inspecciones periódicas, cada 5 años, todas las instalaciones eléctricas en baja tensión que precisaron inspección inicial, según el apartado 4.1 de la ITC BT-05.
    \item Serán objeto de inspecciones periódicas, cada 10 años, todas las instalaciones eléctricas en baja tensión de instalaciones comunes de edificios de viviendas de potencia total instalada superior a 100 kW.
\end{enumerate}
En cualquier caso, estas inspecciones serán realizadas por un Organismo de Control Autorizado, libremente elegido por el titular de la instalación.


\toftagstop{Pliego}
\end{document}