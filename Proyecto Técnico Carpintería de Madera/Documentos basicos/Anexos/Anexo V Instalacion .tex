\documentclass[../main.tex]{subfiles}
\begin{document}
\newpage
\thispagestyle{empty}
\begin{center}
    {\includegraphics[width=0.5\textwidth]{Imagenes/Logo UMA.jpg}\par}
    \vspace{1cm}
    {\bfseries\LARGE \Facultad \par}
    \vspace{0.5cm}
    {\scshape\Large \Grado \par}
    \vspace{1.5cm}
    {\scshape\Huge Anexo V \par}
    \vspace{1.5cm}
    {\scshape\Huge Diseño de Instalación de seguridad contra Incendios \par}
    \vspace{0.5cm}
    {\itshape\Large \TituloProyecto \par}
    \vfill
    {\Large Solicitante: \par}
    {\Large \Solicitante  \par}
    \vspace{1cm}
    {\Large Autores: \par}
    {\Large \Autora \par}
    {\Large \Autor \par}
    \vfill
    {\Large \Fecha \par}
\end{center}


\newpage

\section{Introducción}

Realización del diseñe de la instalación de seguridad contra incendios en la nave industrial, cuyo uso es el de carpintería de madera.

Para su desarrollo se ha empleado el 'Reglamento de seguridad contra incendios en los establecimientos industriales'.

\section{Caracterización del establecimiento industrial}

En el propio edificio, además de realizarse actividades propias a la carpintería (Corte, pulido, barnizado...) se tiene un almacén, una oficina, baños, zona de exposición, vestuario y zona de descanso.

Existen 5 tipos de configuraciones para edificios industriales:

\begin{itemize}
    \item Tipo A: El establecimiento ocupa parcialmente un edificio. En este puede haber otros establecimientos, industriales o no.
    \item Tipo B: El establecimiento ocupa totalmente un edificio. Este está adosado o a una distancia inferior a 3 metros de otro u otros edificios, industriales o no.
    \item Tipo C: El establecimiento ocupa totalmente un edificio, o varios. Está a una distancia mayor de 3 metros de otros edificios y no tiene mercancías combustibles o elementos susceptibles de propagar el incendio.
    \item Tipo D: El establecimiento ocupa un espacio abierto, que puede estar cubierto pero algunas de las fachadas carece de cerramiento lateral
    \item Tipo E: El establecimiento ocupa un espacio abierto, que puede estar parcialmente cubierto y algunas de sus fachadas en la parte cubierta carece de cerramiento lateral.
\end{itemize}

En este caso, la nave industrial está adosada a otras naves industriales y por tanto, su categoría es la B. 

\section{Caracterización por el nivel de riesgo intrínseco}

Para la caracterización del nivel de riesgo, se puede evaluar la carga de fuego, ponderada y corregida en función de su actividad.

Para una actividad de producción, transformación, reparación o cualquier actividad distinta al almacenamiento:
\begin{equation}
    Q_s = \frac{\sum_{1}^{i} \cdot q_{si} \cdot S_i \cdot C_i}{A} \cdot R_a
\end{equation}

Donde:

\begin{itemize}
    \item $Q_s$: densidad de carga de fuego, ponderada y corregida, del sector o área de incendio, en MJ/m2
    \item $q_{si}$: densidad de carga de fuego de cada zona con proceso diferente según los distintos procesos que se realizan, en MJ/m2
    \item $S_i$: superficie de cada zona con proceso diferente y densidad de carga de fuego diferente, en $m^2$
    \item $C_i$: Coeficiente adimensional que pondera el grado de peligrosidad (por la combustibilidad) de cada uno de los combustibles (i) que existen en el sector de incendio
    \item $R_a$: Coeficiente adimensional que corrige el grado de peligrosidad (por la activación) inherente a la actividad industrial que se desarrolla en el sector de incendio, producción, montaje, transformación, reparación, almacenamiento, etc
    \item A: superficie construida del sector de incendio o superficie ocupada del área de incendio, en $m^2$
    
\end{itemize}

Para actividades de almacenamiento:

\begin{equation}
    Q_s = \frac{\sum_{1}^{i} \cdot q_{vi} \cdot C_i \cdot h_i \cdot s_i }{A} \cdot R_a
\end{equation}

Donde:

\begin{itemize}
    \item $q_{vi}$: Carga de fuego, aportada por cada $m^3$ de cada zona con diferente tipo de almacenamiento (i) existente en la zona del incendio.
    \item $h_i$: Altura del almacenamiento de cada uno de los combustibles (i) en metros
    \item $s_i$: Superficie ocupada en planta por cada zona con diferente tipo de almacenamiento (i) existente en el sector de incendio en $m^2$
    \item $Q_s$, $C_i$, $Ra$ y A tienen el mismo significado.
\end{itemize}

Por tanto, es necesario calcular el riesgo intrínseco total de la carpintería. Para ello, es necesario calcular la carga de fuego, ponderada y corregida, con las ecuaciones anteriormente escritas.



\begin{table}[H]
    \centering
    \begin{tabular}{c|c|c|c|c|c|c}
    Actividad & Area [$m^2$] & $q_s$ [MJ/$m^2$] & $C_i$ & Ra & Qs $Q_s$ [MJ/$m^2$]
    \end{tabular}
    \caption{Calculo del riesgo intrínseco del edificio para la fabricación}
\end{table}


\end{document}