\documentclass[11pt]{report}
\usepackage{graphicx}
\usepackage[spanish]{babel}
\usepackage[letterpaper,top = 2cm, bottom = 2cm, left = 3cm, right = 1.5cm, marginparwidth = 1.75cm]{geometry}
\usepackage{subfiles}
\usepackage{subcaption}
\usepackage{lastpage}
\usepackage{fancyhdr}
\usepackage{pdfpages}
\usepackage{tableof}
\usepackage{float}


% Definición de las variables generales del proyecto, se cambian aquí y aparecen los cambios en tos sitios :)
\def\TituloProyecto{Proyecto de diseño de nave industrial para uso como carpintería de madera}
\def\Autora{Mercedes Román Ruiz}
\def\Autor{Javier Galiani Luna}
\def\Asignatura{Oficina Técnica}
\def\Facultad{Escuela de Ingenierías Industriales}
\def\Grado{Grado en Ingeniería Electrónica Industrial}
\def\Solicitante{Rafael Guzmán Sepúlveda}
\def\Fecha{Diciembre 2023}

% Aquí vamos a ir poniendo apuntes generales sobre todo de Latex
% poniendo latex() en matlab te saca el formato matemático en latex :)

%\pagestyle{fancy}
%\fancyhead{}
%\fancyfoot{}
%\fancyfoot[R]{Página \thepage \hspace{1} de \pageref{(LastPage)}}

\setlength{\parindent}{0cm}

\begin{document}
% Cambio de algunos detalles de funciones propias de latex
\renewcommand\thesection{\arabic{section}}
\renewcommand{\baselinestretch}{1.5}
\renewcommand{\listtablename}{Índice de tablas}
\renewcommand{\tablename}{Tabla}


\begin{titlepage}
    \centering
    \includegraphics[width=0.5\textwidth]{Imagenes/Logo UMA.jpg}\par
    \vspace{1cm}
    {\bfseries\LARGE \Facultad \par}
    \vspace{0.5cm}
    {\scshape\Large \Grado \par}
    \vspace{3cm}
    {\itshape\Huge \TituloProyecto \par}
    \vfill
    {\Large Solicitante: \par}
    {\Large \Solicitante \par}
    \vspace{1cm}
    {\Large Autores: \par}
    {\Large \Autora \par}
    {\Large \Autor \par}
    \vfill
    {\Large \Fecha\par}
\end{titlepage}

\subfile{Documentos basicos/Indice General}

\addcontentsline{toc}{part}{Memoria}
\subfile{Documentos basicos/Memoria}

\addcontentsline{toc}{part}{Anexos}
\subfile{Documentos basicos/Anexos}

\addcontentsline{toc}{part}{Planos}
\subfile{Documentos basicos/Planos}

\addcontentsline{toc}{part}{Pliego de Condiciones}
\subfile{Documentos basicos/Pliego de Condiciones}

\addcontentsline{toc}{part}{Estado de Mediciones}
\subfile{Documentos basicos/Estado de Mediciones}

\addcontentsline{toc}{part}{Presupuesto}
\subfile{Documentos basicos/Presupuesto}

\end{document}
