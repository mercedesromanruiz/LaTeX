\chapter{Clasificación y descripción de las máquinas hidráulicas y eólicas}
\section{Clasificación de las máquinas de fluidos}
Se denomina máquina de fluido a todo dispositivo capaz de convertir energía hidráulica en mecánica, o viceversa.

\begin{itemize}
    \item Según la compresibilidad del fluido:
    \begin{itemize}
        \item Hidráulicas: Flujo incompresible
        \item Térmicas: Flujo compresible
    \end{itemize}
    \item Según el sentido de intercambio de energía:
    \begin{itemize}
        \item Motoras: La máquina absorbe energí del fluido para convertirla en energía mecánica en un eje, e.g. turbinas
        \item Generadoras: La máquina aporta energía al fluido, e.g. bombas, ventiladores, compresores
    \end{itemize}
    \item Según el principio de funcionamiento:
    \begin{itemize}
        \item Turbomáquinas: Máquinas rotodinámicas. El intercambio de energía es debido a la variación del momento cinético del fluido al atravesar un elemento móvil denominado rodete o rotor.
        Según la trayectoria del flujo en el rodete:
        \begin{itemize}
            \item Radiales: p.ej. Bomba centrífuga, ventilador centrífugo
            \item Diagonales: p.ej. Turbina Francis
            \item Axiales: p.ej, Turbina Kaplan
            \item Tangenciales: p.ej. Turbina Pelton
        \end{itemize}
        \item Máquinas de desplazamiento positivo: Máquinas volumétricas, e.g. de émbolo, de membrana, de tornillo, rotativas, etc.
    \end{itemize}
\end{itemize}

\section{Bombas rotodinámicas}
Las bombas y ventiladores son Turbomáquinas, Hidráulicas, Generadoras. Su funcionamiento se basa en la Ecuación de Euler de las Turbomáquinas. Estudiiamos las turbobombas o bombas rotodinámicas de mayor uso:
\begin{itemize}
    \item Bombas centrífugas: la dirección del flujo en el rodee es radial.
    \item Bombas axiales: la dirección del flujo en el rodete es axial.
\end{itemize}

\subsection{Clasificación de las bombas rotodinámicas}
\begin{itemize}
    \item Según la dirección del flujo en el rodete:
    \begin{itemize}
        \item Radiales
        \item Diagonales
        \item Axiales
    \end{itemize}
    \item Según el número de escalonamientos: denominamos escalonamiento al conjunto rodete + difusor. Estos permiten aumentar la altura suministrada con el mismo caudal y disminuir el número específico de revoluciones de la máquina.
    \begin{itemize}
        \item Simples o de un escalonamiento
        \item Compuestas o de múltiples escalonamientos
    \end{itemize}
    \item Según la apertura del rodete:
    \begin{itemize}
        \item Cerrado
        \item Doble aspiración
        \item Semiabierto
        \item Abierto
    \end{itemize}
    \item Según el tipo de difusor:
    \begin{itemize}
        \item Difusor con voluta simple
        \item Difusor con doble voluta
        \item Difusor con corona fija con álabes + voluta
    \end{itemize}
    \item Según la posición del eje:
    \begin{itemize}
        \item Horizontal
        \item Vertical
        \item Inclinado
    \end{itemize}
    \item Según la altura o presión suministrada:
    \begin{itemize}
        \item Baja presión: hasta 20-25 mca
        \item Media presión: entre 20 y 60 mca
        \item Alta presión: más de 60 mca
    \end{itemize}
    \item Según el tipo de accionamiento: motor eléctrico, de gasolina, diésel, turbina de gas, etc.
    \item Otras clasificaciones: según el fluido bombeado, los materiales utilizados en su fabricación, el fin al que se destina, etc.
\end{itemize}

\subsection{Elementos constitutivos}
\begin{itemize}
    \item Tubería de aspiración. Lleva el agua a baja presión a la bomba.
    \item Rodete o rotor. Elemento móvil formado por álabes, es donde se produce el intercambio de momento cinético.
    \item Difusor. Formado por una corona difusora de álabes fijos (opcional) y la voluta (también llamada cámara espiral o caracol). Su función es dirigir el flujo de forma ordenada hacia la salida, y recuperar parte de la energía cinética como energía de presión.
    \item Tubería de impulsión. Distribuye el agua a presión por el resto de la instalación.
\end{itemize}

\subsection{Operación de una bonba rotodinámica}
\begin{itemize}
    \item Arranque y parada. En el arranque será necesario cebar la bomba, la parada debe hacerse progresivamente para evitar el golpe de ariete, los cojinetes deben estar suficientemente lubricados, la empaquetadura debe impedir que entre aire durante el funcionamiento de la bomba.
    \item Procedimiento de cebado. Consiste en llenar de agua la tubería de aspiración y el cuerpo de la bomba, para lo cual el aire debe poder escapar al exterior (por la válvula de purgado). Antes de su arranque, la bomba ha de estar completamente llena de líquido. Las bombas rotodinámicas necesitan ser cebadas porque no son autoaspirantes.
\end{itemize}

\subsection{Bombas centrífugas}
Bombas rotodinámicas en donde la dirección del flujo en el rodete es radial. Uso recomendado para grandes saltos de presión y caudales pequeños ($\Delta P$ grande, $\dot{V}$ pequeño).

\subsection{Bombas axiales}
Bombas rotodinámicas en donde la dirección del flujo en el rodete es axial. Utilizadas para grandes caudales y pequeños saltos de presión.

\section{Turbinas hidráulicas}
Las turbinas hidráulicas son Turbomáquinas, Hidráulicas, Motoras. Su funcionamiento se basa en la Ecuación de Euler de las Turbomáquinas. Estudiamos las turbinas de mayor uso:
\begin{itemize}
    \item Turbinas radiales: Francis lentas
    \item Turbinas diagonales: Francis mixtas, Dériaz
    \item Turbinas axiales: Kaplan, hélice, tubulares
    \item Turbinas tangenciales: Pelton, Michell-Banki
\end{itemize}

\subsection{Clasificación de las turbinas hidráulicas}
\begin{itemize}
    \item Según la transformación de la energía en el rodete:
    \begin{itemize}
        \item Turbinas de acción: El intercambio de energía en el rodete es en forma de energía cinética, de modo que no existe cambio de presión entre la entrada y salida. E.g., turbina Pelton.
        \item Turbinas de reacción: Existe un cambio de presión en el fluido al atravesar el rodete. E.g., turbinas Francias y Kaplan.
    \end{itemize}
    \item Grado de reacción: Relaciona la energía de presión respecto a la energía total intercambiada en el rodete.
    \[ \sigma_R = \left( \frac{p}{\rho} \right)^2_1 / \left( \frac{p}{\rho} + \frac{v^2}{2} \right)^2_1 \]
    \item Las turbinas de acción tienen $\sigma_R = 0$ (Pelton, Turgo, Michell-Banki), mientras que las turbinas de reacción tienen $\sigma \neq 0$.
\end{itemize}

\subsection{Turbinas de acción}
El chorro entra y sale de los álabes del rodete (cucharas) a la presión atmosférica, cediendo prácticamente toda su energía y sale del rodete cayendo por su propio peso aguas abajo. Son turbinas de admisión parcial: sólo unos pocos álabes están activos simultáneamente. Regulación de caudal por control del inyector. Cuando existen más de dos inyectores, el eje se sitúa en vertical. La más extendida es la turbina Pelton. Saltos de gran altura (50 - 1800 m), con caudales relativamente reducidos. Potencias desde 100 kW hasta 420 MW. Otras turbinas de acción: Turgo, Michell-Banki.

\subsubsection{Turbina Pelton}
\paragraph{Elemenetos característicos:}
\begin{itemize}
    \item Tobera de inyección: tranforma la altura neta de la turbina en altura cinética del chorro.
    \item Válvula de aguja: regula la potencia de la turbina variando el caudal.
    \item Sistema de regulación de la turbina: pretende mantener la igualdad de los pares motor y resistente (del generador) a fin de mantener un número constante de revoluciones, aunque varía la carga o la altura neta de la turbina.
    \item Caja o carcasa del rodete: para evitar salpicaduras, el rodete se aloja en la carcasa a presión atmosférica.
\end{itemize}

El rodete está formado por álabes en forma de cuchara.
Configuración en eje horizontal (1, 2 ó 3 rodetes en el mismo eje y 1 ó 2 chorros por rodete) y eje vertical (1 rodete con 4 a 6 chorros).

\subsubsection{Otras turbinas de acción}
\begin{itemize}
    \item Turbina Turgo: el chorro incide de forma lateral (9 - 20$^\circ$) en vez de tangencial como en la Pelton. Sus álabes son cucharas simples, de forma elipsoidal.
    \item Turbina Michell-Banki: de flujo transversal o cruzado, rodete en forma de jaula de ardilla.
\end{itemize}

\subsubsection{Comparativa turbinas de acción}
\begin{itemize}
    \item Turgo: más dificil de construir que la Pelton. Sus álabes son más frágiles. A igualdad de potencia, el diámetro es la mitad que el de la Pelton.
    \item Michell-Banki: buenas perspectivas de utilización en mini y micro centrales. Buen rendimiento a cargas parciales.
\end{itemize}

\begin{table}[H]
    \centering
    \resizebox{0.5 \textwidth}{!}{
    \begin{tabular}{| l | l | l | l |}
        \hline
         & Pelton & Turgo & Michell-Banki \\
        \hline
        Salto [m] & 100 - 1900 & 40 - 200 & 5 - 100 \\
        Potencia [MW] & 0,1 - 300 & 0,5 - 10 & 0,75 - 1 \\
        Rendimiento [\%] & 89 - 92 & 80 - 85 & 80 - 81 \\
        \hline
    \end{tabular}
    }
\end{table}

\subsection{Turbinas de reacción}
Son aquellas en las que se da un intercambio de presión en el rodete. Componentes característicos:
\begin{itemize}
    \item Tubería forzada: Lleva el agua a alta presión hasta la turbina, debe estar diseñada para minimizar las pérdidas de carga.
    \item Voluta, cámara espiral o caracol: conduce el fluido hacia el perímetro del distribuidos ganando algo de potencia cinética.
    \item Predistribuidor de álabes fijos: guía el agua de forma eficiente hacia el rodete.
    \item Distribuidor ``Fink'' de álabes móviles: continúa el guiado además de permitir regular el caudal turbinado.
    \item Rodete o rotor: formado por álabes, donde se produce el intercambio de energía cinética.
    \item Tubo difusor: lleva el agua a remanso, permitiendo recuperar algo de energía.
\end{itemize}

\subsubsection{Turbina Francis (J.B. Francis, s. XIX)}
\paragraph{Operación y características:}
Es la turbina hidráulica más empleada en centrales hidroeléctricas debido a su versatilidad, buen rendimiento y funcionamiento probado durante décadas. Cubre alturas de salto entre 30 y 750 metros (e.g. central de Rosshag en Austria con H = 672 m), y potencias de hasta 800 MW. Según su aplicación, el diseño de la turbina Francis varía:
\begin{itemize}
    \item TF radial (lentas): entrada radial al rodete, grandes saltos, pequeña potencia y rendimiento bajo, pero muy económica. Invade el campo de las turbinas Pelton cuando puede competir en costes.
    \item TF diagonal (medias): entrada diagonal al rodete, saltos intermedios, potencia media y buen rendimiento, costes razonables. Es el tipo de turbina Francis más ampliamente utilizada.
    \item TF axial (rápidas): salida axial del rodete, pequeños saltos, elevada potencia y muy buen rendimiento, costes elevados. Invade el campo de las turbinas Kaplan cuando puede competir en costes.
\end{itemize}

\paragraph{Clasificación:}
\begin{itemize}
    \item Por tipo de instalación:
    \begin{itemize}
        \item Cerrada: turbería forzada > cámara espiral > predistribuidor > distribuidor Fink > rodete > tubo de aspiración > canal de descarga.
        \item Abierta o en cámara de agua: cámara de agua (con o sin techo) > rodete > tubo de aspiración > canal de descarga. Para saltos muy pequeños (6 - 10 m) y turbina Francis de muy pequeña potencia (< 1 MW).
    \end{itemize}
    \item Por número de flujos:
    \begin{itemize}
        \item Simples o de un solo flujo: una única admisión de agua al rodete.
        \item Gemelas o de dos flujos: rodete con doble admisión, absorbiendo doble caudal, lo que aumenta la velocidad de firo pudiendo utilizar un alternador más barato (menor número de polos).
    \end{itemize}
    \item Por disposición del eje: Eje horizontal o Eje vertical
    \item Por la altura del salto:
    \begin{itemize}
        \item Alta presión: saltos de más de 80 m.
        \item Baja presión: saltos de menos de 80 m (necesita una caja espiral de grandes dimensiones)
        \item Para una misma potencia, al aumentar la altura, deberá disminuir el caudal de diseño y aumentar la velocidad de giro resultando en una turbina más pequeña.
    \end{itemize}
\end{itemize}

\subsubsection{Turbina Kaplan (Víctor Kaplan, s.XX)}
La turbina Kaplan es una turbina de reacción axial, evolución de la turbina hélice, que presenta álabes orientables. Se emplean para pequeños saltos (10 - 80 m), en un amplio rango de caudales. Buenos rendimienos (~ 95\%), se suelen emplear en centrales de agua fluyente, ya que mantienen mejor rendimiento frente a cargas parciales que las turbinas Francias. Además, a igualdad de potencia, son menos voluminosas.

\subsubsection{Turbina Dériaz}
Turbina semiaxial o de flujo mixto, de reacción. Saltos de mediana altura (20 - 400 m). Potencia unitaria de hasta 300 MW. Se sitúa entre las Francias y las Kaplan, se emplea como grupo reversible bomba-turbina.

\subsubsection{Grupos tubulares}
Son turbinas axiales que carecen de voluta, el rodete es tipo hélice o Kaplan, el distribuido es de forma cónica y tienen forma hidrodinámica. Saltos de pequeña altura (1 - 20 m). Potencias relativamente bajas (100 kW a 40 MW). Subtipos:
\begin{itemize}
    \item Turbina Bulbo: forma hidrodinámica, todo el conjunto está sumergido.
    \item Turbina tipo S: eje del generador se extrae del conjunto, flujo en forma de ``S''.
    \item Turbina Straflo (Straight Flow): polos del generador integrados en el anillo exterior del rodete, eliminando la necesidad de un eje central y permitiendo una construcción más sencilla y económica.
\end{itemize}

\subsubsection{Comparatiba turbinas de reacción}
\begin{itemize}
    \item Kaplan: mejor que las Francias ante vriaciones de carga. Fácil de transportar:
    \item Dériaz: menor volumen de excavación que las Francis. Menor diámetro que las Kaplan.
    \item Bulbo: mejores rendimienos, menor riesfo de cavitación y menor obra civil. Difícil refrigeración del alternador.
\end{itemize}

\begin{table}[H]
    \centering
    \resizebox{0.5 \textwidth}{!}{
    \begin{tabular}{| l | l | l | l | l | l |}
        \hline
         & Francis & Kaplan & Deriaz & Bulbo & Tipo S \\
        \hline
        Salto [m] & 30 - 700 & 6 - 70 & 20 - 400 & 5 - 20 & 3 - 20 \\
        \hline
        Potencia [MW] & 1 - 250 & 20 - 100 & < 300 & 0.1 - 50 & 0.1 - 10 \\
        \hline
        Rendimiento [\%] & 88 - 93 & 80 - 85 & & 91 - 94 & 91 - 94 \\
        \hline
    \end{tabular}
    }
\end{table}

\begin{itemize}
    \item Tubulares (Bulbo) vs Kaplan: se consigue mauor caudal con Bulbo que además tiene menores dimensiones para la misma potencia.
    \item Francis (rápidas) vs Kaplan: Kapan rinde mejor a cargas parciales. Para alturas superiores a 50 m, a mayor potencia, Kaplan trendrá mayor diámetro.
    \item Dériaz vs Kaplan: mayor rendimiento de Dériaz a iguales condiciones de funcionamiento.
    \item Francis vs Dériaz: Dériaz admite más sobrecarga y variaciones de carga, además de tener menor tamaño que Francis.
\end{itemize}

\begin{table}[H]
    \centering
    \resizebox{0.5 \textwidth}{!}{
    \begin{tabular}{| l | l |}
        \hline
        Francis & Pelton \\
        \hline
        Menores dimensiones globales & Menor volumen excavación. \\
        & Infraestructura más sencilla. \\
        \hline
        Menor peso & Más robustas \\
        \hline
        Mayor rendimiento máximo & Mejor rendimiento a carga parcial \\
        \hline
        Aprovecha mayor desnivel & No cavitación \\
        \hline
    \end{tabular}
    }
\end{table}

\section{Aeroturbinas}
\subsection{Clasificación de las turbinas eólicas}
\begin{itemize}
    \item Eje vertical (VAWT) o de arrastre:
    \begin{itemize}
        \item Savonious, Darrieus
        \item Menos extendidas, menor potencia, menor eficiencia
    \end{itemize}
    \item Eje horizontal (HAWT) o de elevación:
    \begin{itemize}
        \item 1, 2 ó 3 palas orientables, con góndola orientable (a sotavento o a barlovento)
        \item El más utilizado es el diseño noruego: 3 palas, eje horizontal, por sustentación, a barlovento. Mayor potencia, mayor eficiencia.
    \end{itemize}
\end{itemize}

\subsection{Operación de las turbinas eólicas}
\begin{itemize}
    \item Velocidad de arranque: la turbina empieza a generar electricidad, $\approx$ 3 - 5 m/s.
    \item Velocidad nominal: mínima velocidad del viento que produce potencia nominal
    \item Velocidad de corte: la turbina se para por motivos de seguridad
    \item Área de barrido: área circular que barren las palas de una trubina de eje horizontal o superficie cilíndrica por las palas de una turbina de eje vertical.
    \item Potencia disponible del viento:
    \[ \dot{W}_{disponible} = \frac{1}{2} \rho v^3 A \]
    \item Densidad disponible del viento:
    \[ \frac{\dot{W}_{disponible}}{A} = \frac{1}{2} \rho v^3 \]
    \item Eficiencia aerodinámica o factor de potencia $C_p$: es la parte de energía que la turbina puede extraer de toda la disponible
    \[ C_p = \frac{\dot{W}_{util}}{\dot{W}_{disponible}} = \frac{\dot{W}_{util}}{\frac{1}{2} \rho v^3 A } \]
    \item Límite de Betz: máximo valor del factor de potencia teórico para una turbina eólica ideal
    \[ C_{p, max} \approx 0.5926\]
\end{itemize}

\subsection{Descripción de las turbinas eólicas}
\begin{itemize}
    \item Rodete: Formado por palas o álabes, solidarias al buje o ``hub''.
    \item Tren de empuje (``Drive train''): Contiene la caja multiplicadora y los ejes del generador.
    \item Góndola (``Nacelle''): Está formada por una carcasa que contiene el tren de empuje, el sistema de orientación, etc.
    \item Sistemas de regulación y control de potencia
    \item Torre
    \item Cimientos
    \item Sistemas eléctricos
\end{itemize}