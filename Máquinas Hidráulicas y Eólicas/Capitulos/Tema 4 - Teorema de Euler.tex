\chapter{Teorema de Euler}
\section{Teoria ideal universal}
\subsection{Método de estudio de las turbomáquinas}
\begin{itemize}
    \item Método puramente experimental. Tanteos al azar en el banco de pruebas de laboratorio $\rightarrow$ lento, inaceptable y anticientifico.
    \item Método matemático rigurosamente exacto. Por resolución de las ecuaciones de Navier-Stokes $\rightarrow$ inabordable en tiempo y recursos.
    \item Método simplificado. Es el método unidimensional, basado en una serie de hipótesis $\rightarrow$ utilizando tres herramientas básicas, podremos abordar cualquier proyecto.
\end{itemize}

\subsection{Hipótesis de partida para TM hidráulicas}
\begin{itemize}
    \item Fluidos no viscosos o fluido ideal: $\mu \approx 0$
    \item Fluido incompresible: $\rho = cte.$
    \item Flujo permanente o estacionario: $\frac{\delta}{\delta t} = 0$
    \item Flujo irrotacional: $\bigtriangledown \times \vec{v} = 0$
\end{itemize}

\subsection{Método de estudio unidimensional}

\section{Triángulos de velocidades}

\subsection{Velocidades del fluido}

Una particula de fluido que entra en un rodete que está girando con velocidad de arrastre $u$ (velocidad tangente al mismo), recorrerá una trayectoria relativa entre los álabes del rodete (conducto) a velocidad relativa $w$ y una trayectoria absoluta (vista por un observador fijo externo al rodete) a velocidad absoluta $v$. De esta manera, definimos:

\begin{itemize}
    \item $u$, como la velocidad de arrastre, que es la velocidad a la que se mueve un punto sólido del rodete.
    \item $w$, como la velocidad relativa, que es la velocidad respecto al observador relativo, que sigue al rodete en su movimiento (como si estuviera sentado en el rodete)
    \item $v$, como la velocidad absoluta, que es la velocidad del fluido con respecto al observador inmóvil o fijo exterior al rodete.
\end{itemize}

\[ \vec{v} = \vec{w} + \vec{u} \]

\section{Teorema de Euler}

\subsection{Primera formulación de la ecuación de Euler}
\[ \dot{W}_B =  \omega \tau_B = \omega \dot{m} (r_2 v_{2t} - r_1 v_{1t}) = \]
\[ = \dot{m} (\omega r_2 v_{2t} - \omega r_1 v_1t) = \dot{m} (u_2 v_{2t} - u_1 v_{1t})\ [W] \]

Primera formulación de la ecuación de Euler, que nos indica que la energía que el fluido transmite al rodete por unidad de caudal másico es:
\[ \frac{\dot{W}_B}{\dot{m}} = u_2 v_{2t} - u_1 v_{1t} \ \left[\frac{J}{kg}\right] \]
Que representa la energía específica intercambiada en el rodete. En el caso de una bommba, es la energía específica absorbida por el fluido en el rodete.

Dividiendo por la gravedad:
\[ \frac{\dot{W}_B}{g \dot{m}} = \frac{1}{g} (u_2 v_{2t} - u_1 v_{1t}) = H_{rodete} \ [m] \]
Que representa la altura absorbida por el fluido.

Multiplicando por la densidad:
\[ \rho \frac{\dot{W}_B}{\dot{m}} = \rho (u_2 v_{2t} - u_1 v_{1t}) =  \rho g H_{rodete} = \Delta p_{rodete} \ \left[ \frac{N}{m^2} \right] \]

\subsection{De la ecuación de Euler podemos deducir...}
\begin{itemize}
    \item La componente tangenete de la velocidad absoluta del fluido es fundamental para el cálculo de la energía específica que el fluido intercambia en el rodete.
    \item La componente normal de la velocidad absoluta (normal a la SCs de entrada y salida del rodete) es necesaria para evaluar el caudal. El caudal en $[m^3/s]$ será igual a esta componente normal de la velocidad absoluta en $[m/s]$ multiplicada por la sección de la SC que atraviesa en $[m^2]$.
    \item Los ángulos $\alpha$ a la entrada y salida del rodete están asociados con la forma del contorno de las tuberías directrices y órganos fijos de las TM.
    \item Los ángulos $\beta$ a la entrada y salida del rodete están asociados con la forma del contorno de los álabes y en general del rodete.
\end{itemize}

\section{Limitaciones de la teoría ideal unidimensional}
Hipótesis para la apliaciónd de la Teoría Ideal Unidimensional de las Turbomáquinas:
\begin{itemize}
    \item Fluido ideal: viscosidad nula y densidad constante.
    \item Régimen permanente: flujo estacionario, propiedades del fluido constantes en cualquier punto del mismo.
    \item FLujo irrotacional: todas las partículas del fluido siguen la misma trayectoria sobre líneas de corriente paralelas entre sí.
    \item Todo sucede como si todas las partículas de fluido entraran y salieran del rodete de la TM con la misma dirección (mismos ángulos de entrada y de salida del rodete para todas las partículas del fluido), los mismos triangulos de velocidad y la misma energía (misma ecuación de Euler).
\end{itemize}

Estas hipótesis son válidas para una máquina centrífuga pura: cuanto más alejada sea la geometría del rodete de la de una turbomáquina centrífuga pura, peor será la aproximación del método unidimensional para su estudio.

\begin{itemize}
    \item Rodete de una máquina centrífuga pura: $u$ y $v_t$ es constante para todas las partículas del fluido a la entrada (en 1) y a la salida (en 2) del rodete respectivamente. Los bordes de entrada y salida de los álabes deben coincidir con superficies transversales (todo el fluido de la sección transversal interacciona con el álabe a la vez) y dichos bordes están a una distancia constante del eje de rotación a fin de que la velocidad $u$ sea la misma para todo el borde.
    \item Rodete radial puto con generatriz de los álabes no paralela al eje de la máquina a la entrada y a la salida: las partículas del fluido no tienen la misma $u$ a la entrada y salida del rodete respectivamente y $v_t$ tampoco es uniforme.
    \item Rodete diagonal on bordes de entrada y salida de los álabes paralelos al rotor de la máquina: las superficies normales a las líneas de corriente no coincicden con los bordes de entrada y salida de los álabes, por lo que notodos los puntos de dichas superficies entran en contacto con el álabes al mismo tiempo y por ello $v_t$ es diferente para los puntos de dichas superficies.
    \item Rodete diagonal con los bordes de entrada y salida de los álabes perpendiculares al zuncho y al cubo: los bordes
\end{itemize}

\section{Problemas}