\chapter{Materiales}
\section{Introducción.}
\subsection{Clasificación global.}
\subsection{Evolución en la ciencia de materiales.}
\subsection{Propiedades mecánicas y no mecánicas.}

\section{Hormigón.}
\subsection{Características tecnológicas.}
\subsection{Características reológicas.}

\subsubsection{Retracción.}
Es la deformación que sufre el hormión a lo largo del tiempo como consecuencia del gradiente de humedades entre el material y el medio ambiente. A la contracción del hormigón como consecuencia de la pérdida de agua se opone la armadura originándose uan fisuración superficial y tensiones remanentes internas. Depende de la humedad relativa, dosificación y grenulometría del hormigón, diámetros de las armaduras y distribución, dimensiones de la pieza,... 

Tipos de retracción:
\begin{itemize}
    \item Por consolidación y segregación.
    \item Plástica: Se produce en el fraguado cuando la velocidad de evaporación del agua supera a la de exudación.
    \item Hidráulica: Se produce después del curado.
\end{itemize}

\subsubsection{Cansancio.}
Es la disminución de la capacidad resistente del hormigón como consecuencia de la aplicación de cargas lentas en comparación al valor que se obtiene ante cargas rápidas.

\subsection{Características mecánicas.}
% Comportamiento a compresión para cargas instantáneas

% Comportamiento a compresión hormigón

% Propiedades de cálculo para hormigón armado

\subsubsection{Resistencia a tracción.}
Normalmente se refiere a la resistencia característica. Las expresiones más usadas son:
\begin{itemize}
    \item Característico: $f_{ct, k} = 0,21 \cdot \sqrt[3]{f_{ck}^2}$
    \item Medio: $f_{ct, m} = 0,30 \cdot \sqrt[3]{f_{ck}^2}$
    \item Cuantil $95\%$: $f_{ct, k 0,95} = 0,39 \cdot \sqrt[3]{f_{ck}^2}$
\end{itemize}
Escasa aplicación en proyecto.

\subsubsection{Rigidez.}
Al no ser el hormigón un cuerpo  elástico, más 

\subsection{Especificación.}

\section{Aceros estructurales.}
\subsection{Tipos.}
\begin{itemize}
    \item Perfiles laminados en caliente: Es el tipo más empleado en construcción. Se obtienen transformando el acero en bruto a alta temperaturra usando trenes de laminación. Se agrupan en series por la forma y características de su sección transversal ()
    \item Perfiles conformados en frío: Se fabrican mediante conformadoras de rodillo en frío a partir de chapas finas de acero. El conformado en frío les confiere unas características específicas en lo que se refiere a la sección y a la resistencia mecánica.
    \item Cables: Usados en puentes atirantados y colgantes y cubiertas de grandes luces. Se suele usar acero trefilado, que proporciona alta resistencia- Este acero se
    \item Tornillos:
    \item Otros:
\end{itemize}

\subsection{Características mecánicas.}
Ley de comportamiento de un acero de dureza natural.

\begin{itemize}
    \item Límite elástico: Condiciona estados límites (alargamientos inaceptables de barras).
    \item Tensión de rotura: Condición estados límite (rotura de secciones).
    \item Endurecimiento: Permite el alargamiento de las barras.
    \item Alargamiento en rotura: Permite la redistribución de esfuerzos.
    \item Resistencia a rotura frágil: Condición previa a cualquier comprobación.
\end{itemize}

Ductilidad: Capacidad del acero de alcanzar grandes deformaciones sin llegar a la fractura.
Cuanto mayor sea el área encerrada por la zona plástica en el diagrama tensión-deformación, mayor será la ductilidad.
Los aceros dúctiles tienen mejor comportamiento a fratiga frente a cagas cíclicas.
Esta propiedad es muy importante en el proyecto sísmico de estructuras.

\section{Armaduras.}
\subsection{Tipos.}
\begin{itemize}
    \item Armaduras parivas:
    \item Armaduras activas: Son las que se emplean en el hormmigón pretensado (tnedones o barras). Resisten las cargas activamente. Sólo se emplean aceros de alta resistencia.
\end{itemize}

\subsection{Adherencia.}
La adherencia hormigón-acero es fundamental para que el hormigón armado pueda funcionar como material estructural. Mediante la adherencia se asegura el anclaje de las barras de acero y se controla la disuración del hormigón.

Los mecanismos de adherencia son de naturaleza físico-química (provocan la adhesión del acero con el hormigón a través de fuerzas capilares desarrolladas en la interfase) o mecánica (la penetración del cemento en las irregularidades de las barras provoca un efecto de rozamiento que favorece la adherencia y, además, en el caso de barras corrugadas, las corrugas hacen un efecto de cuña que aumenta la resistencia al deslizamiento).

\section{Otros materiales de construcción.}

\section{Suelos}
\subsection{Generalidades.}
\begin{itemize}
    \item Rocas: masas minerales sin forma constante
    \item suelos: resultado alteración de las Rocas
    \item Materia orgánica
    \item Particularidades de suelos. El sulo presenta características muy distintas de otros materiales:
    \begin{itemize}
        \item No es posible su elección
        \item No se trata de un material homogéneo (aunque se intente asimilar).
        \item El contenido en agua tiene gran importancia en su comportamiento.
    \end{itemize}
\end{itemize}

\subsection{Clasificación.}
Arenas: alteración física de las rocas. Se caracterizan por no tener resistencia a tracción y sus propiedades dependen principalmente del tamaño de  partículas y su compacidad. Se clasifican mediante curvas granulométricas (tamizado de las muestras).

Arcillas: alteración química de las rocas. Son una mezcla de geles amorfos y partículas de especies mineralógicas generalmente con forma laminar. Sus propiedades dependen principalmente de la composición y humedad. Se clasifican mediante los límites de atterberg (plasticidad frente a humedad).

\subsection{Índices de fases.}
Suelo formado por la mezcla de tres fases (sólido, líquido y gas).
\begin{itemize}
    \item Suelos secos: huecos están llenos de aire.
    \item Suelos saturados: huecos están llenos de agua.
    \item Suelos parcialmente saturados: contienen aire y agua.
\end{itemize}

El agua en el suelo:
\begin{itemize}
    \item Agua Freática (acuíferos) - Nivel freático vs base rocosa
    \item Agua por encima del nivel freático:
    \begin{itemize}
        \item Agua higroscópica - fuerzas atracción
        \item Agua absorbida - tensión superficial
        \item Agua capilar - asciende de la capa freática
    \end{itemize}
\end{itemize}

\subsection{Propiedades elementales.}
\begin{itemize}
    \item Pesos específicos:
    \begin{itemize}
        \item Peso específico aparente:
        \item Peso específico seco:
        \item Pero específico saturado:
        \item Peso específico sumergido:
        \item Peso específico fase sólida
    \end{itemize}
    \item Índices de fase:
    \begin{itemize}
        \item Índice de huecos:
        \item Porosidad:
        \item Humedad:
        \item Saturación:
    \end{itemize}
\end{itemize}

\subsection{Principio de Terzaghi o Principio de las tensiones efectivas.}
Representación dle efcecto del agua en el suelo:
\begin{itemize}
    \item Equilibrio: tensiones totales
    \item Comportamiento: tensiones efectivas (no totales)
\end{itemize}
\[\sigma'_{ij} = \sigma_{ij} - u \cdot \delta_{ij}\]