\chapter{Materiales}
\section{Introducción.}
\subsection{Clasificación global.}
\subsection{Evolución en la ciencia de materiales.}
\subsection{Propiedades mecánicas y no mecánicas.}

\section{Hormigón.}
\subsection{Características tecnológicas.}
\subsection{Características reológicas.}

\subsubsection{Retracción.}
Es la deformación que sufre el hormión a lo largo del tiempo como consecuencia del gradiente de humedades entre el material y el medio ambiente. A la contracción del hormigón como consecuencia de la pérdida de agua se opone la armadura originándose uan fisuración superficial y tensiones remanentes internas. Depende de la humedad relativa, dosificación y grenulometría del hormigón, diámetros de las armaduras y distribución, dimensiones de la pieza,... 

Tipos de retracción:
\begin{itemize}
    \item Por consolidación y segregación.
    \item Plástica: Se produce en el fraguado cuando la velocidad de evaporación del agua supera a la de exudación.
    \item Hidráulica: Se produce después del curado.
\end{itemize}

\subsubsection{Cansancio.}
Es la disminución de la capacidad resistente del hormigón como consecuencia de la aplicación de cargas lentas en comparación al valor que se obtiene ante cargas rápidas.

\subsection{Características mecánicas.}

\subsection{Especificación.}

\section{Aceros estructurales.}

\section{Armaduras.}

\section{Otros materiales de construcción.}

\section{Suelos}
