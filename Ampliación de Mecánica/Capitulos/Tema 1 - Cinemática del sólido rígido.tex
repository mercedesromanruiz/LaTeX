\chapter{Cinemática	del sólido rígido}
\section{Ecuaciones}
\subsection{Velocidades y aceleraciones}
Dada una referencia cartesiana $\{O, \hat{x}, \hat{y}, \hat{z}\}$, podemos expresar el vector de posición de un punto $P$, mediante la expresión:
\[ \vec{r}_P = \vec{OP} = x \hat{x} + y \hat{y} + z \hat{z} \]
su velocidad:
\[ \vec{v}_P = \dot{x} \hat{x} + \dot{y} \hat{y} + \dot{z} \hat{z} \]
y su aceleración:
\[ \vec{a}_P = \ddot{x} \hat{x} + \ddot{y} \hat{y} + \ddot{z} \hat{z} \]

\subsection{Movimiento de un sólido rígido}
Sea un sólido rígido $\sigma$ que experimenta una transformación desde $t_0$ hasta $t$. En este proceso, las distancias entre dos puntos cualesquiera de $\sigma$ no cambian. Matemáticamente:
\[ \frac{d \vec{AB} \cdot \vec{AB}}{dt} = 0 \rightarrow 2 \vec{AB} \cdot \frac{d \vec{AB}}{dt} = 0 \]

Además, si denotamos la transformación por $M(t)$, es claro que se trata de una transformación lineal sobre los vectores que unen dos puntos cualesquiera de $\sigma$:
\[ S(t) (\lambda \vec{AB}) = \lambda S(t) (\vec{AB}) \]
\[ S(t) (\vec{AD} + \vec{DE} = S(t) (\vec{AD})) + S(t) (\vec{DE}) \]

\subsection{Teorema de la rotación de Euler}

\subsection{Vector axial de rotación}

\subsection{Sólido rígido}

\subsection{Torsor cinemático}

\section{Cinemática relativa}
\subsection{Velocidades}

\subsection{Vectores}

\subsection{Aceleraciones}

\section{Composición de movimientos}
\subsection{Tres sistemas}

\subsection{Generalización}