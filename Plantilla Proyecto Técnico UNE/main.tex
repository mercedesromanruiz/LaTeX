% --------------------- COPYRIGHT --------------------

% Plantilla realizada por: Mercedes Román Ruiz
% Contacto: mercedesromanruiz@gmail.com

% ----------------------------------------------------

\documentclass[11pt]{report}

\usepackage[spanish]{babel} % Definir el idioma del documento
\usepackage[letterpaper,top = 2cm, bottom = 2cm, left = 3cm, right = 1.5cm, marginparwidth = 1.75cm]{geometry} % Especificar los márgenes según la norma
\usepackage{pdfpages} % Añadir PDFs y que te cuente las páginas para el documento
\usepackage{subfiles} % Tener subdocumentos (hay más opciones)
\usepackage{tableof} % Se utiliza para los indices de los subdocumentos
\usepackage{float} % Para usar lo de 'H'

\def\TituloProyecto{Título del Proyecto Técnico}
\def\Autor{Nombre y Apellidos}
\def\Solicitante{Nombre y Apellidos}
\def\Fecha{Mes AAAA}

% ---------------------------- LOS TRUQUITOS -------------------------------------------------
% Si se incluye [H] la tabla, grafica o imagen en el sitio en el que se está escribiendo, sin importar las sugerencias de LaTeX
% --------------------------------------------------------------------------------------------

\begin{document}

\renewcommand\thesection{\arabic{section}}
\renewcommand{\baselinestretch}{1.5}
\renewcommand{\listtablename}{Índice de tablas} % Si no se hace esto, aparece como 'Indice de Cuadros'
\renewcommand{\tablename}{Tabla} % Si no se hace esto, aparece como 'Cuadro'

\begin{titlepage}
    \centering
    \includegraphics[width=0.25\linewidth]{LogoEjemplo.png} \par
    \vspace{3 cm}
    {\itshape\Huge \TituloProyecto \par}
    \vfill
    {\Large Solicitante:  \Solicitante \par}
    \vspace{0.5cm}
    {\Large Autor:  \Autor \par}
    \vspace{1.5cm}
    {\Large \Fecha \par}
\end{titlepage}

\subfile{Documentos Basicos/Indice General}

\addcontentsline{toc}{part}{Memoria}
\subfile{Documentos Basicos/Memoria}

\addcontentsline{toc}{part}{Anexos}
\subfile{Documentos Basicos/Anexos}

\addcontentsline{toc}{part}{Planos}
\subfile{Documentos Basicos/Planos}

\addcontentsline{toc}{part}{Pliego de Condiciones}
\subfile{Documentos Basicos/Pliego de Condiciones}

\addcontentsline{toc}{part}{Estado de Mediciones}
\subfile{Documentos Basicos/Estado de Mediciones}

\addcontentsline{toc}{part}{Presupuesto}
\subfile{Documentos Basicos/Presupuesto}

\end{document}