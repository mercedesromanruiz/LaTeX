\documentclass[main]{subfiles}

\begin{document}
\newpage
\thispagestyle{empty}
\begin{center}
    \centering
    \includegraphics[width=0.25\linewidth]{LogoEjemplo.png} \par
    \vspace{3 cm}
    {\scshape\Huge Anexos \par}
    \vspace{1.5cm}
    {\itshape\Huge \TituloProyecto \par}
    \vfill
    {\Large Solicitante:  \Solicitante \par}
    \vspace{0.5cm}
    {\Large Autor:  \Autor \par}
    \vspace{1.5cm}
    {\Large \Fecha \par}
\end{center}

% El documento básico Anexos se iniciará con un índice que hará referencia a cada uno de los documentos, a sus capítulos y apartados que los componen, con el fin de facilitar su utilización.
\chapter*{Índice de Anexos:}
\tableof{Anexos}
\newpage
\toftagstart{Anexos}

% Está formado por los documentos que desarrollan, justifican o aclaran apartados específicos de la memoria u otros documentos básicos del Proyecto. Este documento contendrá los anexos necesarios.

% ----------------------- EJEMPLO ADJUNTAR ANEXOS ------------------------------------
%\addcontentsline{toc}{chapter}{Anexo I: Fontanería}
%\subfile{Anexos/Anexo I Instalacion de Fontaneria}
% ------------------------------------------------------------------------------------

\toftagstop{Anexos}
\end{document}