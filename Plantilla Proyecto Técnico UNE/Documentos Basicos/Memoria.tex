\documentclass[main]{subfiles}

\begin{document}
\newpage
\thispagestyle{empty}
\begin{center}
    \centering
    \includegraphics[width=0.25\linewidth]{LogoEjemplo.png} \par
    \vspace{3 cm}
    {\scshape\Huge Memoria \par}
    \vspace{1.5cm}
    {\itshape\Huge \TituloProyecto \par}
    \vfill
    {\Large Solicitante:  \Solicitante \par}
    \vspace{0.5cm}
    {\Large Autor:  \Autor \par}
    \vspace{1.5cm}
    {\Large \Fecha \par}
\end{center}

% El Estado de Mediciones se iniciará con un índice que hará referencia a cada uno de los documentos, a sus capítulos y apartados que los componen, con el fin de facilitar su utilización.

\chapter*{Índice de la Memoria:}
\tableof{Memoria}
\newpage
\toftagstart{Memoria}

\section{Objeto}
% En este capítulo de la Memoria se indicrá el objetivo del Proyecto y su justificación.

\section{Alcance}
% En este capítulo de la Memoria se indicará el ámbito de aplicación del Proyecto.

\section{Antecedentes}
% En este capítulo de la Memoria se enumerarán todos aquellos aspectos necesarios para la comprensión de las alternativas estudiadas, y la solución final adoptada.

\section{Normas y referencias}
% En este capítulo de la Memoria se relacionarán sólo los documentos citados en los distintos apartados de la misma.

\subsection{Disposiciones legales y normas aplicadas}
% En este apartado se contemplará el conjunto de disposiciones legales (leyes, reglamentos, ordenanzas, etc.) y las normas de no obligado cumplimiento que se han tenido en cuenta para la realización del Proyecto.

\subsection{Bibliografía}
% En este apartado se contemplará el conjunto de libros, revistas u otros textos que el autor considere de interés para justificar las soluciones adoptadas en el Proyecto.

\subsection{Programas de cálculo}
% En este apartado se contemplará la relación de progrmas, modelos y otras herramientas utilizadas para desarrollar los diversos cálculos del Proyecto.

\subsection{Plan de gestión de la calidad aplicado durante la redacción del Proyecto}
% En este apartado se enunciarán los procesos específicos utilizados para asegurar la calidad durante la realización del Proyecto.

\subsection{Otras referencias}
% En este apartado se incluirán aquellas referencias que, no estando relacionadas en los apartados anteriores, se consideren de interés para la comprensión y materialización del Proyecto.

\section{Definiciones y abreviaturas}
% En este capitulo de la Memoria se relacionarán todas las definiciones, abreviaturas, etc. que se han utilizado y su significado.

\section{Requisitos de diseño}
% En este capítulo de la Memoria se describirán las bases y datos de partida establecidos.

\section{Análisis de soluciones}
% En este capítulo de la Memoria se indicarán las distintas alternativas estudiadas, qué caminos se han seguido para llegar a ellas, ventajas e inconvenientes de cada una y cuál es la solución finalmente elegida y su justificación.

\section{Resultados finales}
% En este capítulo de la Memoria se describirá el producto, obra, instalación, servicio o software (soporte lógico) según la solución elegida, indicando cuáles son sus características definitorias y haciendo referencia a los planos y otros elementos del Proyecto que lo definen.

\section{Planificación}
% En este capítulo de la Memoria, y en relación al proceso de materialización del objeto del Proyecto, se definirán las diferentes etapas, metas o hitos a alcanzar, plazos de entrega y cronogramas o gráficos de programación correspondientes.

\section{Orden de prioridad entre los documentos básicos}
% En este capítulo de la Memoria el autor del Proyecto, frente a posibles discrepancias, deberá estableces el orden de prioridad de los documentos básicos del Proyecto. Si no se especifica, el orden de prioridad será el siguiente: Planos - Pliego de Condiciones - Presupuesto - Memoria

\toftagstop{Memoria}
\end{document}