\documentclass[main]{subfiles}

\begin{document}
\newpage
\thispagestyle{empty}
\begin{center}
    \centering
    \includegraphics[width=0.25\linewidth]{LogoEjemplo.png} \par
    \vspace{3 cm}
    {\scshape\Huge Presupuesto \par}
    \vspace{1.5cm}
    {\itshape\Huge \TituloProyecto \par}
    \vfill
    {\Large Solicitante:  \Solicitante \par}
    \vspace{0.5cm}
    {\Large Autor:  \Autor \par}
    \vspace{1.5cm}
    {\Large \Fecha \par}
\end{center}

% El Presupuesto se iniciará con un índice que hará referencia a cada uno de los documentos, a sus capítulos y apartados que los componen, con el fin de facilitar su utilización.

\chapter*{Índice del Presupuesto:}
\tableof{Presupuesto}
\toftagstart{Presupuesto}

% El Presupuesto contendrá:
% Cuadro de precios unitarios de materiales, mano de obra y elementos auxiliares que componen las partidas o unidades de obra.
% Un cuadro de precios unitarios de las unidades de obra, de acuerdo con el Estado de Mediciones y con la descomposición correspondiente de materiales, mano de obra y elementos auxiliares.
% El presupuesto propiamente dicho que contendrá la valoración económica global, desglosada y ordenada según el Estado de mediciones.


\toftagstop{Presupuesto}
\end{document}