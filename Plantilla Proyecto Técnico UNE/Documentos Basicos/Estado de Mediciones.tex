\documentclass[main]{subfiles}

\begin{document}
\newpage
\thispagestyle{empty}
\begin{center}
    \centering
    \includegraphics[width=0.25\linewidth]{LogoEjemplo.png} \par
    \vspace{3 cm}
    {\scshape\Huge Estado de Mediciones \par}
    \vspace{1.5cm}
    {\itshape\Huge \TituloProyecto \par}
    \vfill
    {\Large Solicitante:  \Solicitante \par}
    \vspace{0.5cm}
    {\Large Autor:  \Autor \par}
    \vspace{1.5cm}
    {\Large \Fecha \par}
\end{center}

% El Estado de Mediciones se iniciará con un índice que hará referencia a cada uno de los documentos, a sus capítulos y apartados que los componen, con el fin de facilitar su utilización.

\chapter*{Índice de Estado de Mediciones:}
\tableof{Mediciones}
\newpage
\toftagstart{Mediciones}

% Contendrá un listado completo de las partidas de obra que configuran la totalidad del Proyecto. Se subdivirá en distintos apartados o subapartados, correspondientes a las partes más significativas del objeto del Proyecto. Servirá de base para la realización del Presupuesto.

\toftagstop{Mediciones}
\end{document}