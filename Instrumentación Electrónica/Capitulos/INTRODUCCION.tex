\chapter{Introducción al resumen}
Como se puede observar en el índice, hay capítulos o secciones repetidas. Esto se debe al orden que sigue el resumen. Cada capítulo es una pregunta típica de examen, el contenido del capítulo se ha desarrollado a partir de los libros de referencia del profesor (sí, esos que pone al final de la guía docente que nadie se lee) y sus diapositivas de clase. 

El orden de los capítulos no es al azar, fui preguntando y recolectando preguntas de años anteriores (2020, 2021, 2022 y 2023) y lo puse todo en un excel (no estoy loca, lo prometo). Quedando así:

\begin{table}[H]
    \centering
    \begin{tabular}{|c|c|}
        \hline
        Tema & Veces Preguntadas en 8 Examenes \\
        \hline
        Puente de Wheatstone & 7 \\
        Amplificador de Instrumentación & 6 \\
        Cálculo del Ch del S\&H & 5 \\
        Sensor Monolítico de Temperatura & 3 \\
        Problema del Interruptor Abierto / Cerrado & 2 \\
        Convertidores A/D & 2 \\
        Sensores & 2 \\
        Convertidor Doble Rampa & 2 \\
        Sensores de Presión & 2 \\
        Convertidor de Aproximaciones Sucesivas & 2 \\
        Convertidores D/A & 1 \\
        Sensores de Desplazamiento, Velocidad y Aceleración & 1 \\
        Convertidor Simple Rampa & 1 \\
        Termistor & 1 \\
        Filtros & 1 \\
        Sensores de Temperatura & 1 \\
        \hline
    \end{tabular}
\end{table}