\chapter{Energías renovables: Solar}
\section{Introducción.}
Del Sol llega a la superficie terrestre una potencia equivalente a la producida por $\sim 100$ millones de centrales nucleares. La cuestión es ¿cómo aprovecharla?

\section{Recurso Solar}
Se define la constante solar como la energía procedente del Sol, recibida por unidad de tiempo sobre la unidad de área de una superficie perpendicular a la radiación extraterrestre (que no ha sufrido ninguna atenuación atmosférica) en una distancia media anual Tierra-Sol.

Teniendo en cuenta la variación de la distancia Tierra-Sol, la irradiancia se corrige como:
\begin{equation}
    G_{ON} = G_{SC} \cdot (1 + 0,033 \cdot cos \frac{360 \cdot n}{365})
\end{equation}
Donde $G_ {SC} = 1353\ W/m^2$

Cuanto mayor sea la distancia recorrida por la radiación en la atmósfera mayor será la probabilidad de que se produzca procesos de absorción y dispersión.

Formas de aprovechamiento de la energía solar que llega a la superficie de la Tierra:
\begin{itemize}
    \item Heliotérmica: aprovecha la radiación térmica tal cual, para generar focos caloríficos.
    \begin{itemize}
        \item Solar Térmica: uso directo del calor generado por la radiación solar
        \item Solar Termoeléctrica: generación de electricidad a partir del calor generado por la radiación solar
    \end{itemize}
    \item Fotovoltaica: aprovecha algunos fotones suficientemente energéticos para crear pares electrón/hueco en un semiconductor y generar de esta forma una corriente eléctrica.
\end{itemize}

\section{Energía Solar Térmica}
Consiste en la transformación de la energía solar en energía térmica:
\begin{itemize}
    \item Transformación en baja temperatura $(35-90^\circ C)$: Paneles solares. Uso doméstico:
    \begin{itemize}
        \item Instalaciones de A.C.S.
        \item Instalaciones de calefacción
        \item Instalaciones de refrigeración
    \end{itemize}
    \item Transformación en media temperatura $(90-400^\circ C)$: Uso industrial. Bajo-medio índice de concentración.
    \begin{itemize}
        \item Colectores cilindro-parabólico
        \item Colectores lineales Fresnel
    \end{itemize}
    \item Transformación en alta temperatura $(>400^\circ C)$. Centrales termoeléctricas. Alto índice de concentración.
    \begin{itemize}
        \item Helióstatos de alta concentración
        \item Discos parabólicos
    \end{itemize}
\end{itemize}

Características:
\begin{itemize}
    \item Aplicaciones de baja temperatura
    \item Aprovecha rradiación directa y difusa
    \item No requiere seguimiento de la trayectoria solar
    \item Escaso mantenimiento
    \item Más sencillas que las instalaciones con colector de concentración
\end{itemize}

El Código Técnico de la Edificación obliga desde 2007 a que toda vivienda de nueva construcción conste de aporte de energías renovables, siendo el más habitual Energía Solar de apoyo a ACS.

La importancia del producto ``transmitancia-absortancia'' estriba en la posibilidad de recoger en un factor unificado los efectos