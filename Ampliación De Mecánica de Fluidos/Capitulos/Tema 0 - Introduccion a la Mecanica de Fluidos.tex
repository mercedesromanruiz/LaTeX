\chapter{Introducción a la Mecánica de Fluidos}
\section{Diferencia entre sólido y fluido}
Según su estructura interna:
\begin{itemize}
    \item Sólido: A nivel microscópico una estructura molecular ordenada. Red cristalina.
    \item Fluido: Líquidos y gases.
    \begin{itemize}
        \item Estructura molecular no definida.
        \item Fuerzas intermoleculares más débiles que los sólidos.
        \item Tienen una gran capacidad de deformación. Adoptan la forma del recipiente que lo contiene.
    \end{itemize}
\end{itemize}

Según su deformación por acción de fuerzas externas:
\begin{itemize}
    \item Tensiones normales: Responden con una deformación proporcional a la tensión aplicada. Siendo la deformación más grande en fluidos que en sólidos.
    \item Tensiones tangenciales:
    \begin{itemize}
        \item Un solido responde con una deformación estática proporcional a la fuerza aplicada.
        \item Un fluido se deforma de manera indefinida, con una velocidad de deformación proporcional a la fuerza aplicada.
    \end{itemize}
\end{itemize}

Definición de Fluido: Sustancia que no soporta tangenciales en equilibrio.

\subsection{Reología}
Es la ciencia que trata de establecer relaciones entre las tensiones aplicadas a los cuerpos y sus deformaciones. 

Existen sustancias que tienen un comportamiento híbrido entre sólido y fluido. Según la intensidad de las tensiones tangenciales; y según la frecuencia de la aplicación dinámica de la fuerza.

En realidad todo fluye en la Naturaleza, solo hay que esperar el tiempo suficiente para observarlo.

Número de Deborah (De) es un número adimensional usado en reología para lo ``fluido'' que es un material.

\section{Diferencia entre líquidos y gases}
La clave es la densidad: $\rho(T, p)$. $\rho_{liq} >> \rho_{gas}$

\section{Hipótesis de Medio Continuo}
A escala macroscópica podemos asumir que la materia está distribuida de forma continua (sin huecos).
