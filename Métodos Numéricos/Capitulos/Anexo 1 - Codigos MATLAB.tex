\onecolumn
\chapter*{Códigos MATLAB}
\section*{Raíces de Ecuaciones y Sistemas No Lineales}

\begin{lstlisting}[caption={Método de la Bisección}, language = Matlab, label = {cod: Método de la Bisección}]
clear all;
% Datos
syms x;
f = @(x) exp(x) - sin(x);       % Funcion
a = -4; b = -3;                 % Intervalo [a, b]
N = 100;                        % Numero maximo de iteraciones
tol = 10^-6;                    % Tolerancia

u = f(a);
v = f(b);
e = b - a;

for i = 1 : N
    e = e/2;
    c = a + e;
    w = f(c);
    if abs(e) < tol
        break;
    end
    if sign(w) ~= sign(u)
        b = c;
        v = w;
    else
        a = c;
        u = w;
    end
    fprintf(['i: ', num2str(i), ' c:' , num2str(c), ' f(c):', num2str(f(c)), '\n'])
end
\end{lstlisting}

\begin{lstlisting}[caption={Iteración de Punto Fijo}, language = Matlab, label = {cod: Iteración de Punto Fijo}]
    clear all;
    % Datos
    syms x;
    f = @(x) cos(x) - x;    % Funcion
    g = @(x) cos(x);        % Despejamos x de la funcion
    p0 = pi / 4;            % Aproximacion inicial p0
    tol = 10e-3;            % Tolerancia
    N = 100;                % Numero maximo de iteraciones
    
    % Algoritmo
    for i = 1 : N
        p = g(p0);
        if abs(p - p0) < tol
            round(p, 4)
            break;
        end
        p0 = p;
    end
\end{lstlisting}

\begin{lstlisting}[caption={Método de Newton}, language = Matlab, label = {cod: Método de Newton}]
    clear all;
    % Datos
    syms x;
    f = @(x) cos(x) - x;    % Funcion
    p0 = 0.5;               % Aproximacion inicial p0
    tol = 10e-3;            % Tolerancia
    N = 100;                % Numero maximo de iteraciones
    
    % Algoritmo
    for i = 1 : N
        p = p0 - f(p0) / subs(diff(f, x), p0);
        if abs(p - p0) < tol
            round(p, 4)     % El procedimiento fue exitoso
            break;
        end
        p0 = p;
    end
\end{lstlisting}

\begin{lstlisting}[caption={Método de la Secante}, language = Matlab, label = {cod: Método de la Secante}]
    clear all;
    % Datos
    syms x;
    f = @(x) cos(x) - x;    % Funcion
    p0 = 0.5;               % Aproximacion inicial p0
    p1 = pi / 4;            % Aproximacion inicial p1
    tol = 10e-3;            % Tolerancia
    N = 100;                % Numero maximo de iteraciones
    
    % Algoritmo
    q0 = f(p0);
    q1 = f(p1);
    for i = 1 : N
        p = p1 - q1 * (p1 - p0) / (q1 - q0);
        if abs(p - p1) < tol
            round(p, 4)     % El procedimiento fue exitoso
            break;
        end
        p0 = p1;
        q0 = q1;
        p1 = p;
        q1 = f(p);
    end
\end{lstlisting}

\begin{lstlisting}[caption={Método de la Posición Falsa}, language = Matlab, label = {cod: Método de la Posición Falsa}]
    clear all;
    % Datos
    syms x;
    f = @(x) cos(x) - x;    % Funcion
    p0 = 0.5;               % Aproximacion inicial p0
    p1 = pi / 4;            % Aproximacion inicial p1
    tol = 10e-3;            % Tolerancia
    N = 100;                % Numero maximo de iteraciones
    
    % Algoritmo
    q0 = f(p0);
    q1 = f(p1);
    for i = 1 : N
        p = p1 - q1 * (p1 - p0) / (q1 - q0);
        if abs(p - p1) < tol
            round(p, 4)     % El procedimiento fue exitoso
            break;
        end
        q = f(p);
        if q * q1 < 0
            p0 = p1;
            q0 = q1;
        end
        p1 = p;
        q1 = q;
    end
\end{lstlisting}

\begin{lstlisting}[caption={Método de Newton para Sistemas}, language = Matlab, label = {cod: Método de Newton para Sistemas}]
    clear all;
    format long;
    % Datos
    syms x1 x2 x3;
    f = [3*x1 - cos(x2 * x3) - 1/2;
         x1^2 - 81*(x2 + 0.1)^2 + sin(x3) + 1.06;
         exp(-x1*x2) + 20*x3 + (10*pi - 3) / 3];
    x0 = [0.1; 0.1; -0.1];                              % Aproximacion inicial
    N = 100;
    tol = 10e-5;
    j = jacobian(f);
    
    % Algoritmo
    x = x0;
    y = [];
    for i = 1 : 100
        F = subs(f, [x1, x2, x3], [x(1), x(2), x(3)]);
        J = subs(j, [x1, x2, x3], [x(1), x(2), x(3)]);
        y = -J\F;
        y = round(y, 10);       % Si esto no se pone puede que se quede colgado el bucle
        if norm(y) < tol
            break;
        end
        x = x + y;
    end
    
    x
\end{lstlisting}

\section*{Optimización Con Restricciones}

\begin{lstlisting}[caption={Método del Penalty}, language = Matlab]
    syms x y a
    F(x,y) = 100 * (y - x^2)^2 + (1 - x)^2;           % Funcion
    f = matlabFunction(F, 'Vars', [x,y]);
    
    H(x,y) = (x + 0.5)^2 + (y + 0.5)^2 - 0.25;        % Restricciones
    h = matlabFunction(H, 'Vars', [x,y]);
    
    L(x,y,a) = F(x,y) + a/2 * (H(x,y)^2);             % Lagrangiano penalizado
    
    gL = gradient(L, [x,y]);
    gl = matlabFunction(gL, 'Vars', [x,y,a]);
    
    hL = hessian(L, [x,y]);
    hl = matlabFunction(hL, 'Vars', [x,y,a]);
    
    z = [1; 1];                                       % Punto de Inicio
    x = z(1); y = z(2);
    a = 0.3;                                          % Factor de Penalizacion Inicial NO MUY GRANDE
    b = 4;                                            % Tiene que estar entre 4 y 10
    
    nit = 0; grad = [0;0]; Hess = zeros(2); solx = []; soly = [];
    V = [h(x,y) 0];
    M = max(V);
    tol = 1e-8;
    while M > tol
        rho = 0.1; beta = 0.5;
        maxit = 100;
        for i = 1 : maxit
            grad = gl(x,y,a);
            Hess = hl(x,y,a);
            d = -Hess\grad;
            if abs(d' * grad) < tol
                break;
            end
            alpha = 1; 
            for k = 1 : 50   
                znew = z + alpha * d;
                xnew = znew(1);
                ynew = znew(2);
                Lznew = L(xnew, ynew, a);
                if Lznew < L(x, y, a) + alpha * rho * grad' * d
                    break;
                else 
                    alpha = alpha * beta;
                end
            end
            z = z + alpha * d;
            x = z(1); y = z(2);
        end
    
        nit = nit + 1;
        solx(nit) = x;
        soly(nit) = y;
        V = [h(x, y) 0];
        M = max(V);
        a = a * b;
    end
    
    u = -2.048 : 0.1 : 2.048; 
    v = -2.048 : 0.1 : 2.048;
    [A, B] = meshgrid(u, v);
    
    figure;
    hold on
    contour(A, B, f(A,B))
    fimplicit(h,[-2.048 2.048])
    plot(solx,soly,'*')
    hold off
\end{lstlisting}

\begin{lstlisting}[caption={Método del Lagrangiano Aumentado}, language = Matlab]
    clear all;
    syms x y a l;
    F(x,y) = 100 * (y - x^2)^2 + (1 - x)^2;           % Funcion
    f = matlabFunction(F, 'Vars', [x,y]);
    
    H(x,y) = (x + 0.5)^2 + (y + 0.5)^2 - 0.25;        % Restricciones
    h = matlabFunction(H, 'Vars', [x,y]);
    
    L(x,y,l,a) = F(x,y) - l * H(x,y) + 1/2 * a * (H(x,y)^2);             % Lagrangiano aumentado
    
    gL = gradient(L, [x,y]);
    gl = matlabFunction(gL, 'Vars', [x,y,l,a]);
    
    hL = hessian(L, [x,y]);
    hl = matlabFunction(hL, 'Vars', [x,y,l,a]);
    
    z = [4; 2];                                       % Punto de Inicio
    x = z(1); y = z(2);
    a = 0.3;                                          % Factor de Penalizacion Inicial NO MUY GRANDE
    b = 4;                                            % Tiene que estar entre 4 y 10
    l0 = 15;                                          % Primera Iteracion Multiplicador de Lagrange
    
    nit = 0; grad = [0;0]; Hess = zeros(2); solx = []; soly = [];
    V = [h(x,y) 0];
    M = max(V);
    tol = 1e-8;
    while M > tol
        l0 = l0 - a * h(x,y);
        rho = 0.1; beta = 0.5;
        maxit = 100;
        for i = 1 : maxit
            grad = gl(x,y,l0,a);
            Hess = hl(x,y,l0,a);
            d = -Hess\grad;
            if abs(d' * grad) < tol
                break;
            end
            alpha = 1; 
            for k = 1 : 50   
                znew = z + alpha * d;
                xnew = znew(1);
                ynew = znew(2);
                Lznew = L(xnew, ynew, l0, a);
                if Lznew < L(x, y, l0, a) + alpha * rho * grad' * d
                    break;
                else 
                    alpha = alpha * beta;
                end
            end
            z = z + alpha * d;
            x = z(1); y = z(2);
        end
    
        nit = nit + 1;
        solx(nit) = x;
        soly(nit) = y;
        V = [h(x, y) 0];
        M = max(V);
        a = a * b;
    end
    
    u = -2.048 : 0.1 : 2.048; 
    v = -2.048 : 0.1 : 2.048;
    [A, B] = meshgrid(u, v);
    
    figure;
    hold on
    contour(A, B, f(A,B))
    fimplicit(h,[-2.048 2.048])
    plot(solx,soly,'*')
    hold off
\end{lstlisting}

\section*{Métodos Para Ecuaciones Diferenciales Ordinarias}
\begin{lstlisting}[caption={Método de Euler}, language = Matlab]
    clear all;
    format long;
    
    syms t y;
    f = @(t, y) y - t^2 + 1;            % y' = y - t^2 + 1
    a = 0;   b = 2;                     % 0 <= t <= 2
    N = 10;                             % Numero Maximo de Iteraciones
    alpha = 0.5;                        % y(0) = 0.5    Condicion Inicial
    
    h = (b - a)/N;
    t = a;
    w = alpha;
    
    for i = 1 : N
        w = w + h * f(t, w);
        t = a + i * h;
    end
    
    t = zeros(1, N + 1);    t(1) = a;
    w = zeros(1, N + 1);    w(1) = alpha;
    
    for i = 1 : N
        w(i + 1) = w(i) + h * f(t(i), w(i));
        t(i + 1) = a + i * h;
    end
    
    y = (t + 1).^2 - 0.5 * exp(t);      % Valores Exactos  
    error = abs(y - w);                 % Error Cometido
    
    figure;
    plot(t, y, 'b-', 'DisplayName', 'Valores Exactos: y');
    hold on;
    plot(t, w, 'r--', 'DisplayName', 'Aproximacion: w');
    title('PVI : Metodo de Euler');
    xlabel('t');
    ylabel('w | y')
    legend('show')
    hold off;
\end{lstlisting}

\begin{lstlisting}[caption={Método de Runge-Kutta (orden 4)}, language = Matlab]
    clear all;
    format long;
    
    syms t y;
    f = @(t, y) y - t^2 + 1;        % Funcion
    a = 0;   b = 2;                 % Intervalo [a, b]
    alpha = 0.5;                    % Condicion inicial y(a) = alpha
    N = 10;                         % Numero de Iteraciones
    
    h = (b - a) / N;
    t = zeros(1, N + 1);        t = a;
    w = zeros(1, N + 1);        w = alpha;
    
    for i = 1 : N
        K1 = h * f(t(i), w(i));
        K2 = h * f(t(i) + h/2, w(i) + K1/2);
        K3 = h * f(t(i) + h/2, w(i) + K2/2);
        K4 = h * f(t(i) + h, w(i) + K3);
    
        w(i + 1) = w(i) + (K1 + 2 * K2 + 2 * K3 + K4)/6;
        t(i + 1) = a + i * h;
    end
    
    y = (t + 1).^2 - 0.5 * exp(t);      % Valores Exactos  
    error = abs(y - w);                 % Error Cometido
    
    figure;
    plot(t, y, 'b-', 'DisplayName', 'Valores Exactos: y');
    hold on;
    plot(t, w, 'r--', 'DisplayName', 'Aproximacion: w');
    title('PVI : Metodo de Runge-Kutta (orden 4)');
    xlabel('t');
    ylabel('w | y')
    legend('show')
    hold off;
\end{lstlisting}
