\chapter{Diferenciación Numérica}

\section{Diferenciación Numérica}
Ya hemos visto una forma de aproximar la derivada de una función $f$:
\begin{equation}
    f'(x) = \frac{f(x + h) - f(x)}{h}
    \label{Eq: (9.1)}
\end{equation}
para algún número pequeño $h$. Para determinar la precisión de esta aproximación, utilizamos el teorema de Taylor, asumiendo que $f \in C^2$: 
\[ f(x + h) = f(x) + h f'(x) + \frac{h^2}{2} f''(\xi) \quad \xi \in [x, x+h] \rightarrow \]
\[ f'(x) = \frac{f(x + h) - f(x)}{h} - \frac{h}{2} f''(\xi)\]

El término $\frac{h}{2} f''(\xi)$ se llama error de truncamiento, o error de discretización, y se dice que la aproximación es de primer orden ya que el error de truncamiento es $O(h)$.

Sin embargo, el redondeo también juega un papel en la evaluación del cociente de diferencias finitas (\ref{Eq: (9.1)}). Por ejemplo, si $h$ es tan pequeño que $x + h$ se redondea a $x$, entonces el cociente de diferencia calculado será 0. Más generalmente, incluso si el único error que se comete es en el redondeo de los valores $f(x + h)$ y $f(x)$, el cociente de diferencia calculado será
\[ \frac{f(x + h) (1 + \delta_1) - f(x) (1 + \delta_2)}{h} = \]
\[ = \frac{f(x + mh) - f(x)}{h} + \frac{\delta_1 f(x + h) - \delta_2 f(x)}{h}\]

Dado que cada $|\delta_i|$ es menor que la precisión de la máquina $\epsilon$, esto implica que el error de redondeo es menor o igual a
\[ \frac{\epsilon(|f(x)| + |f(x + h)|)}{h} \] 

Dado que el error de truncamiento es proporcional a $h$ y el error de redondeo es proporcional a $1/h$, la mejor precisión se alcanza cuando estas dos cantidades son aproximadamente iguales. Ignorando las constantes $|f''(\xi)/2|$ y ($|f(x)| + |f(x + h)|$), esto significa que
\[ h \approx \frac{\epsilon}{h} \rightarrow h \approx \sqrt{\epsilon}\]
y en este caso el error (error de truncamiento o error de redondeo) es aproximadamente $\sqrt{\epsilon}$. Así, con la fórmula (\ref{Eq: (9.1)}), podemos aproximar una derivada solo a aproximadamente la raíz cuadrada de la precisión de la máquina.

Otra forma de aproximar la derivada es utilizar una fórmula de diferencia centrada:
\begin{equation}
    f'(x) = \frac{f(x + h) - f(x - h)}{2h}
    \label{Eq: (9.2)}
\end{equation}

Podemos nuevamente determinar el error de truncamiento utilizando el teorema de Taylor. Expandiendo $f(x + h)$ y $f(x - h)$ alrededor del punto $x$, encontramos
\[ f(x + h) = f(x) + h f'(x) + \frac{h^2}{2} f''(x) + \frac{h^3}{6} f'''(\xi) \]
\[ \xi \in [x, x+h] \]
\[ f(x - h) = f(x) - h f'(x) + \frac{h^2}{2} f''(x) - \frac{h^3}{6} f'''(\eta) \]
\[ \eta \in [x-h, x] \]

Restando las dos ecuaciones y resolviendo para $f'(x)$ se obtiene
\[ f'(x) = \frac{f(x + h) - f(x - h)}{2h} - \frac{h^2}{12} (f'''(\xi) + f'''(\eta)) \]
Por lo tanto, el error de truncamiento es $O(h^2)$, y esta fórmula de diferencia es de segundo orden.

Para estudiar los efectos del redondeo, nuevamente hacemos la suposición simplificadora de que el único redondeo que ocurre es en el redondeo de los valores $f(x + h)$ y $f(x - h)$. Entonces, el cociente de diferencia calculado es
\[ \frac{f(x + h) (1 + \delta_1) - f(x - h)(1 + \delta_2)}{2h} = \]
\[ = \frac{f(x + h) - f(x - h)}{2h} + \frac{\delta_1 f(x + h) - \delta_2 f(x - h)}{2h} \]
y el término de redondeo $(\delta_1 f(x + h) - \delta_2 f(x - h)) / 2h$ está acotado en valor absoluto por $\epsilon (\delta_1 f(x + h) - \delta_2 f(x - h)) / (2h)$. Nuevamente, ignorando los términos constantes que involucran a $f$ y sus derivadas, la mayor precisión se logra cuando
\[ h^2 \approx \frac{\epsilon}{h} \rightarrow h \approx \epsilon^{1/3} \]
y el error (error de truncamiento o error de redondeo) es $\epsilon^{2/3}$. Con esta fórmula, podemos obtener mayor precisión, hasta aproximadamente la potencia $2/3$ de la precisión de la máquina.

Se puede aproximar derivadas de orden superior de manera similar. Para derivar una aproximación de segundo orden a la segunda derivada, nuevamente utilizamos el teorema de Taylor:
\[ f(x + h)  = f(x) + h f'(x) + \frac{h^2}{2!} f''(x) + \frac{h^3}{3!} f'''(x) + \frac{h^4}{4!} f''''(\xi) \]
\[ \xi \in [x, x+h] \]
\[ f(x - h) = f(x) - h f'(x) + \frac{h^2}{2!} f''(x) - \frac{h^3}{3!} f'''(x) + \frac{h^4}{4!} f''''(\eta) \]
\[ \eta \in [x - h, x] \]

Sumando estas dos ecuaciones se obtiene
\[ f(x + h) + f(x - h) = 2 f(x) + h^2 f''(x) + \frac{h^4}{12} f''''(v) \]
\[ v \in [\eta, \xi] \]

Resolviendo para $f''(x)$, obtenemos la fórmula
\[ f''(x) = \frac{f(x + h) - 2 f(x) + f(x - h)}{h^2} - \frac{h^2}{12} f''''(v)\]

Usando la aproximación
\begin{equation}
    f''(x) = \frac{f(x + h) - 2 f(x) + f(x - h)}{h^2}
    \label{Eq: (9.3)}
\end{equation}
el error de truncamiento es $O(h^2)$. Sin embargo, note que un análisis similar del error de redondeo predice un error de redondeo de tamaño $\epsilon / h^2$, por lo que el menor error total ocurre cuando $h$ es aproximadamente $\epsilon^{1/4}$ y luego el error de truncamiento y el error de redondeo son cada uno aproximadamente $\sqrt{\epsilon}$. Con una precisión de máquina $\epsilon \approx 10^{-16}$, esto significa que $h$ no debe ser tomado como menor que aproximadamente $10^{-4}$. La evaluación de cocientes estándar de diferencias finitas para derivadas de orden superior es aún más sensible a los efectos del redondeo.

\subsection{Formulas de tres puntos}
\subsubsection{Fórmula de tres puntos hacia delante}
\begin{equation}
    f'(x) = \frac{-3 f(x) + 4 f(x + h) - f(x + 2h)}{2h}  + O(h^2)
\end{equation}

\subsubsection{Fórmula de tres puntos hacia atrás}
\begin{equation}
    f'(x) = \frac{f(x - 2h) - 4 f(x - h) + 3 f(x)}{2h} + O(h^2)
\end{equation}

\subsubsection{Fórmula de tres puntos centrada}
\begin{equation}
    f'(x) = \frac{-f(x - h) + f(x + h)}{2h} - O(h^2)
\end{equation}

\subsection{Fórmulas de Cinco Puntos}
Una fórmula común de cinco puntos se utiliza para determinar aproximaciones de la derivada en el punto medio.

\subsubsection{Fórmula de Cinco Puntos en el Punto Medio}
\begin{multline}
    f'(x) = \frac{1}{12h} [ f(x - 2h) - 8f(x - h) + 8 f(x + h) \\
    - f(x + 2h) ] + O(h^4)
\end{multline}

\subsubsection{Fórmula de Cinco Puntos en el Extremo}
\begin{multline}
    f'(x) = \frac{1}{12h} [ -25f(x) + 48 f(x + h) - 36 f(x + 2h) \\
    + 16 f(x + 3h) - 3 f(x + 4h) ] + O(h^4)
\end{multline}

\subsection{Inestabilidad del Error de Redondeo}
Es particularmente importante prestar atención al error de redondeo al aproximar derivadas. Para ilustrar esta situación, examinemos más de cerca la fórmula de tres puntos en el punto medio,
\[ f'(x_0) = \frac{1}{2h} [f(x_0 + h) - f(x_0 - h)] - \frac{h^2}{6} f^{(3)}(\xi_1) \]
Supongamos que al evaluar $f(x_0 + h)$ y $f(x_0 - h)$ encontramos errores de redondeo $e(x_0 + h)$ y $e(x_0 - h)$. Entonces, nuestros cálculos realmente utilizan los valores $\tilde{f}(x_0 + h)$ y $\tilde{f}(x_0 - h)$, que están relacionados con los valores verdaderos $f(x_0 + h)$ y $f(x_0 - h)$ por $ f(x_0 + h) = \tilde{f}(x_0 + h) + e(x_0 + h) $ y $ f(x_0 - h) = \tilde{f}(x_0 - h) + e(x_0 - h) $.

El error total en la aproximación,
\[ f'(x_0) - \frac{\tilde{f}(x_0 + h) - \tilde{f}(x_0 - h)}{2h} = \]
\[ = \frac{e(x_0 + h) - e (x_0 - h)}{2h} - \frac{h^2}{6} f^{(3)}(\xi_1)\]
es debido tanto al error de redondeo, la primera parte, como al error de truncamiento. Si asumimos que los errores de redondeo $e(x_0 \pm h)$ están acotados por un número $\epsilon > 0$ y que la tercera derivada de $f$ está acotada por un número $M > 0$, entonces 
\[ \left| f'(x_0) - \frac{\tilde{f}(x_0 + h) - \tilde{f}(x_0 - h)}{2h} \right| \leq \frac{\epsilon}{h} + \frac{h^2}{6}M\]

Para reducir el error de truncamiento, $h^2 M/6$, necesitamos reducir $h$. Pero al reducir $h$, el error de redondeo $\epsilon/h$ crece. En la práctica, por lo tanto, rara vez es ventajoso hacer que $h$ sea demasiado pequeño, porque en ese caso el error de redondeo dominará los cálculos.

\section*{Ejercicios}
\noindent (1) Reproducir todos los ejemplos del Greenbaum sección 9.1

\noindent (2) Utilziar el Teorema de Taylor para hallar la siguiente fórmula de diferenciación numérica, dejando indicado el orden del error de truncamiento del problema.
\[ f'(x) \approx \frac{1}{2h} (-3f(x) + 4f(x + h) - f(x + 2h)) \]
\noindent (3) Para la fórmula de aproximación de la derivada segunda
\[ f''(x) \approx \frac{f(x + h) - 2f(x) + f(x -h)}{h^2} \]
determinar el tamño de paso $h$, en función de $\epsilon_M$ a partir del cual el error cometido al aproximar $f''$ vuelve a crecer. Determinar el orden de ese error en función de $\epsilon_M$.

\noindent (4) Utilizar el Teorema de Taylor para hallar la siguiente fórmula de diferenciación numérica, dejando indicado el orden del error de truncamiento del problema.
\[ f'''(x) \approx \frac{f(x + 2h) - f(x + h) + 2f(x - h) - f(x - 2h)}{2h^3} \]

\noindent (5) Utilizar el Teorema de Taylor para determinar los coeficientes $A, B, C$ y $D$ de la fórmula
\[ f''(x) \approx \frac{Af(x + 3h) + Bf(x + 2h) + Cf(x + h) + Df(x)}{h^2} \]
e indicar el orden del error de truncamiento.