\chapter{Aparamenta eléctrica de baja tensión}
\section{Introducción}
\begin{enumerate}
    \item Circuitos eléctricos. Todo circuito eléctrico, trifásico o monofásico, debe llevar en su origen un aparato capaz de conectarlo o desconectarlo. Ahora bien, la conexión o desconexión (maniobra) del circuito puede hacerse en diferentes condiciones de funcionamiento del circuito, lo que condicionará el tipo de elemento de maniobra que se vaya a colocar.
    \item Condiciones de funcionamiento de los circuitos eléctricos. Un circuito eléctrico se puede maniobrar estando en las siguientes condiciones:
    \begin{itemize}
        \item Vacío. Es decir, sin cargas conectadas, por lo que la corriente se corta (al abrir) o se establece (al cerrar) es nula ($I = 0$). La tensión del circuito es la nominal ($U \approx U_N$).
        \item Funcionamiento normal. $U = U_N$; $I = I_N$ Hay cargas conectadas y la corriente que se corta o se presenta en el circuito (al abrir o cerrar) es igual o menor que la corriente nominal del circuito. Se denomina corriente nominal a la máxima que puede circular permanentemente por el circuito, sin producir calentamientos excesivos ni ningún problema de otro tipo.
        \item Funcionamiento en carga anormal. Al efectuar la maniobra la corriente que se conecta o desconecta es superior a la nominal. La presencia de corrientes anormales puede ser debida al exceso de carga, al funcionamiento defectuoso de alguna de las cargas alimentadas por el circuito o a averías en el propio circuito. Estas corrientes anormales pueden ser:
        \begin{itemize}
            \item Sobrecargas, que son corriente cuyo valor es algo superior al nominal ($I_N < I < 10 \cdot I_N$)
            \item Cortocircuitos, corrientes de valor muy superior al nominal ($I_N \geq 10 \cdot I_N$) originadas por averías graves en las cargas o en los cables de la instalación; normalmente son fallos de aislamieento.
        \end{itemize}
    \end{itemize}
    \item Ideas básicas sobre el arco eléctrico. 
    En la mayor parte de los dispositivos estudiados en este capítulo, la apertura o cierre del circuito en el que están instalados se realiza por la separación o unión de unas piezas metálicas de gran conductividad llamadas contactos.

    Cuando un aparato corta o establece una corriente aparece un arco eléctrico entre sus contactos. En general, se presenta un arco eléctrico entre dos partes conductoras con niveles de tensión diferentes, separadas por un medio aislante, cuando el valor del campo eléctrico en algún punto del aislante supera a su rigidez dieléctrica, entonces se ioniza el medio aislante produciéndose el arco.

    En el arco se disipa una gran cantidad de energía, se alcanzan altas temperaturas y se volatilizan partes pequeñas del material de los contactos. Desde el punto de vista eléctrico, el arco se comporta como una resistencia de valor variable.

    Un aparato eléctrico que conecta o desconecta circuitos debe eliminar (extinguir) muy rápidamente el arco eléctrico que se presenta para evitar que la energía liberada lo deteriore. Realmente, los arcos más problemáticos se producen durante la apertura de los contactos. Durante el cierre el arco se extingue por sí mismo, produciéndose arcos adicionales en el caso de que haya rebotes de los contactos tras el primer cierre.

    Si el circuito es de corriente continua, el aparato debe dispone de los medios para extinguir (apagar) el arco. Si el aparato es de corriente alterna, cuando la corriente pasa por cero el arco se extingue, pero el medio está ionizado y a alta temperatura, debido a la energía disipada, por lo que el aparato debe disponer de los medios para evitar que el arco se reencienda tras los primeros pasos por cero de la corriente.

    \item Características básicas de los aparatos de corte.
    Los aparatos de maniobra deben diseñarse para la tensión de circuito donde se van a instalar y para el mayor valor de la corriente que permanentemente pued pasar pos sus contactos en estado de cerrados sin provocar calentamientos excesivos ni averías e el
\end{enumerate}
\subsection{Definiciones y características de la aparamente eléctrica}
\subsection{Magnitudes de definición comunes a la aparamenta de conexión}

\section{Interruptores automáticos}
\subsection{Disparadores}
\subsection{Características de los interruptores automáticos}

\section{Pequeños Interruptores Automáticos (PIA)}

\section{Fusibles}
\subsection{Principio de funcionamiento de los fusibles}
\subsection{Denominación de los fusibles}
\subsection{Fusibles limitadores}
\subsection{Aplicaciones de los fusibles}

\section{Contactores}
\subsection{Clases de contactores}
\subsection{Características de los contactores}
\subsection{Contactores con semiconductores}
\subsection{Aplicaciones del contactor}

\section{Interruptores y relés diferenciales}
\subsection{Fundamentos de la protección diferencial}
\subsection{Sistema de detección}
\subsection{Interruptor diferencial}
\subsection{Transformador y relé diferencial}
\subsection{Aplicaciones de la protección diferencial}