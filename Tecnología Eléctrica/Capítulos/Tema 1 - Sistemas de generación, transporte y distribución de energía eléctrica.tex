\chapter{Sistemas de generación, transporte y distribución de energía eléctrica}
\section{Introducción}
En este capítulo se ofrece una visión general del sistema eléctrico en su conjunto, desde la generación de la energía eléctrica hasta su distribución alos usuarios finales.

\section{Centrales generadores}
Una central productors de energía eléctrica (central eléctrica) es una instalación destinada a transformar energía de cualquier otro tipo en energía eléctrica.

La mayoría de centrales utiliza la energía mecánica de algún fluido (agua, vapor, viento) para mover una máquina (turbina, motor térmico, hélice, etc.), que a su vez arrastra a un generador eléctrico. En la actualidad, casi todos los generadores que se utilizan son máquinas síncronas que generan un sistema trifásico de tensiones senosoidales.

El generador síncrono, también llamado alternador, produce una tensión senoidal sobre una bobina (inducido), haciendo girar en su interior un campo magnético que se produce en el inductor, el cual puede asimilarse a un imás, aunque en general se trata de un electroimán. La parte móvil de la máquina, que contiene el inductor, se llama rotor y la parte fija qu elo envuelve se llama estatos, y en él se ubican las bobinas del inducido.

A continuación se comentan, de una forma somera, algunas características de los principales tipos de centrales eléctricas.

Centrales hidráulicas. Entre las primer